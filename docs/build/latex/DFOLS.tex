%% Generated by Sphinx.
\def\sphinxdocclass{report}
\documentclass[letterpaper,10pt,english]{sphinxmanual}
\ifdefined\pdfpxdimen
   \let\sphinxpxdimen\pdfpxdimen\else\newdimen\sphinxpxdimen
\fi \sphinxpxdimen=.75bp\relax
\ifdefined\pdfimageresolution
    \pdfimageresolution= \numexpr \dimexpr1in\relax/\sphinxpxdimen\relax
\fi
%% let collapsible pdf bookmarks panel have high depth per default
\PassOptionsToPackage{bookmarksdepth=5}{hyperref}

\PassOptionsToPackage{booktabs}{sphinx}
\PassOptionsToPackage{colorrows}{sphinx}

\PassOptionsToPackage{warn}{textcomp}
\usepackage[utf8]{inputenc}
\ifdefined\DeclareUnicodeCharacter
% support both utf8 and utf8x syntaxes
  \ifdefined\DeclareUnicodeCharacterAsOptional
    \def\sphinxDUC#1{\DeclareUnicodeCharacter{"#1}}
  \else
    \let\sphinxDUC\DeclareUnicodeCharacter
  \fi
  \sphinxDUC{00A0}{\nobreakspace}
  \sphinxDUC{2500}{\sphinxunichar{2500}}
  \sphinxDUC{2502}{\sphinxunichar{2502}}
  \sphinxDUC{2514}{\sphinxunichar{2514}}
  \sphinxDUC{251C}{\sphinxunichar{251C}}
  \sphinxDUC{2572}{\textbackslash}
\fi
\usepackage{cmap}
\usepackage[T1]{fontenc}
\usepackage{amsmath,amssymb,amstext}
\usepackage{babel}



\usepackage{tgtermes}
\usepackage{tgheros}
\renewcommand{\ttdefault}{txtt}



\usepackage[Bjarne]{fncychap}
\usepackage{sphinx}

\fvset{fontsize=auto}
\usepackage{geometry}


% Include hyperref last.
\usepackage{hyperref}
% Fix anchor placement for figures with captions.
\usepackage{hypcap}% it must be loaded after hyperref.
% Set up styles of URL: it should be placed after hyperref.
\urlstyle{same}

\addto\captionsenglish{\renewcommand{\contentsname}{Contents:}}

\usepackage{sphinxmessages}
\setcounter{tocdepth}{1}



\title{DFO-LS Documentation}
\date{30 October 2024}
\release{1.5.3}
\author{Lindon Roberts}
\newcommand{\sphinxlogo}{\vbox{}}
\renewcommand{\releasename}{Release}
\makeindex
\begin{document}

\ifdefined\shorthandoff
  \ifnum\catcode`\=\string=\active\shorthandoff{=}\fi
  \ifnum\catcode`\"=\active\shorthandoff{"}\fi
\fi

\pagestyle{empty}
\sphinxmaketitle
\pagestyle{plain}
\sphinxtableofcontents
\pagestyle{normal}
\phantomsection\label{\detokenize{index::doc}}


\sphinxAtStartPar
\sphinxstylestrong{Release:} 1.5.3

\sphinxAtStartPar
\sphinxstylestrong{Date:} 30 October 2024

\sphinxAtStartPar
\sphinxstylestrong{Author:} \sphinxhref{https://lindonroberts.github.io/}{Lindon Roberts}

\sphinxAtStartPar
DFO\sphinxhyphen{}LS is a flexible package for finding local solutions to nonlinear least\sphinxhyphen{}squares minimization problems (with optional regularizer and constraints), without requiring any derivatives of the objective. DFO\sphinxhyphen{}LS stands for Derivative\sphinxhyphen{}Free Optimizer for Least\sphinxhyphen{}Squares.

\sphinxAtStartPar
That is, DFO\sphinxhyphen{}LS solves
\begin{equation*}
\begin{split}\min_{x\in\mathbb{R}^n}  &\quad  f(x) := \sum_{i=1}^{m}r_{i}(x)^2 + h(x) \\
\text{s.t.} &\quad  a \leq x \leq b\\
            &\quad x \in C := C_1 \cap \cdots \cap C_n, \quad \text{all $C_i$ convex}\\\end{split}
\end{equation*}
\sphinxAtStartPar
The optional regularizer \(h(x)\) is a Lipschitz continuous and convex, but possibly non\sphinxhyphen{}differentiable function that is typically used to avoid overfitting.
A common choice is \(h(x)=\lambda \|x\|_1\) (called L1 regularization or LASSO) for \(\lambda>0\).
Note that in the case of Tikhonov regularization/ridge regression, \(h(x)=\lambda\|x\|_2^2\) is not Lipschitz continuous, so should instead be incorporated by adding an extra term into the least\sphinxhyphen{}squares sum, \(r_{m+1}(x)=\sqrt{\lambda} \|x\|_2\).
The (optional) constraint set \(C\) is the intersection of multiple convex sets provided as input by the user. All constraints are non\sphinxhyphen{}relaxable (i.e. DFO\sphinxhyphen{}LS will never ask to evaluate a point that is not feasible), although the general constraints \(x\in C\) may be slightly violated from rounding errors.

\sphinxAtStartPar
Full details of the DFO\sphinxhyphen{}LS algorithm are given in our papers:
\begin{enumerate}
\sphinxsetlistlabels{\arabic}{enumi}{enumii}{}{.}%
\item {} \begin{enumerate}
\sphinxsetlistlabels{\Alph}{enumii}{enumiii}{}{.}%
\setcounter{enumii}{2}
\item {} 
\sphinxAtStartPar
Cartis, J. Fiala, B. Marteau and L. Roberts, \sphinxhref{https://doi.org/10.1145/3338517}{Improving the Flexibility and Robustness of Model\sphinxhyphen{}Based Derivative\sphinxhyphen{}Free Optimization Solvers}, \sphinxstyleemphasis{ACM Transactions on Mathematical Software}, 45:3 (2019), pp. 32:1\sphinxhyphen{}32:41 {[}\sphinxhref{https://arxiv.org/abs/1804.00154}{preprint}{]} .

\end{enumerate}

\item {} \begin{enumerate}
\sphinxsetlistlabels{\Alph}{enumii}{enumiii}{}{.}%
\setcounter{enumii}{12}
\item {} 
\sphinxAtStartPar
Hough, and L. Roberts, \sphinxhref{https://doi.org/10.1137/21M1460971}{Model\sphinxhyphen{}Based Derivative\sphinxhyphen{}Free Methods for Convex\sphinxhyphen{}Constrained Optimization}, \sphinxstyleemphasis{SIAM Journal on Optimization}, 21:4 (2022), pp. 2552\sphinxhyphen{}2579 {[}\sphinxhref{https://arxiv.org/abs/2111.05443}{preprint}{]}.

\end{enumerate}

\item {} \begin{enumerate}
\sphinxsetlistlabels{\Alph}{enumii}{enumiii}{}{.}%
\setcounter{enumii}{24}
\item {} 
\sphinxAtStartPar
Liu, K. H. Lam and L. Roberts, \sphinxhref{http://arxiv.org/abs/2407.14915}{Black\sphinxhyphen{}box Optimization Algorithms for Regularized Least\sphinxhyphen{}squares Problems}, \sphinxstyleemphasis{arXiv preprint arXiv:arXiv:2407.14915}, 2024.

\end{enumerate}

\end{enumerate}

\sphinxAtStartPar
DFO\sphinxhyphen{}LS is a more flexible version of \sphinxhref{https://github.com/numericalalgorithmsgroup/dfogn}{DFO\sphinxhyphen{}GN}.

\sphinxAtStartPar
If you are interested in solving general optimization problems (without a least\sphinxhyphen{}squares structure), you may wish to try \sphinxhref{https://github.com/numericalalgorithmsgroup/pybobyqa}{Py\sphinxhyphen{}BOBYQA}, which has many of the same features as DFO\sphinxhyphen{}LS.

\sphinxAtStartPar
DFO\sphinxhyphen{}LS is released under the GNU General Public License. Please \sphinxhref{http://www.nag.com/content/worldwide-contact-information}{contact NAG} for alternative licensing.

\sphinxstepscope


\chapter{Installing DFO\sphinxhyphen{}LS}
\label{\detokenize{install:installing-dfo-ls}}\label{\detokenize{install::doc}}

\section{Requirements}
\label{\detokenize{install:requirements}}
\sphinxAtStartPar
DFO\sphinxhyphen{}LS requires the following software to be installed:
\begin{itemize}
\item {} 
\sphinxAtStartPar
Python 3.9 or higher (\sphinxurl{http://www.python.org/})

\end{itemize}

\sphinxAtStartPar
Additionally, the following python packages should be installed (these will be installed automatically if using \sphinxstyleemphasis{pip}, see {\hyperref[\detokenize{install:installation-using-pip}]{\sphinxcrossref{Installation using pip}}}):
\begin{itemize}
\item {} 
\sphinxAtStartPar
NumPy (\sphinxurl{http://www.numpy.org/})

\item {} 
\sphinxAtStartPar
SciPy version 1.11 or higher (\sphinxurl{http://www.scipy.org/})

\item {} 
\sphinxAtStartPar
Pandas (\sphinxurl{http://pandas.pydata.org/})

\end{itemize}

\sphinxAtStartPar
\sphinxstylestrong{Optional package:} DFO\sphinxhyphen{}LS versions 1.2 and higher also support the \sphinxhref{https://github.com/lindonroberts/trust-region}{trustregion} package for fast trust\sphinxhyphen{}region subproblem solutions. To install this, make sure you have a Fortran compiler (e.g. \sphinxhref{https://gcc.gnu.org/wiki/GFortran}{gfortran}) and NumPy installed, then run \sphinxcode{\sphinxupquote{pip install trustregion}}. You do not have to have trustregion installed for DFO\sphinxhyphen{}LS to work, and it is not installed by default.


\section{Installation using conda}
\label{\detokenize{install:installation-using-conda}}
\sphinxAtStartPar
DFO\sphinxhyphen{}LS can be directly installed in Anaconda environments using \sphinxhref{https://anaconda.org/conda-forge/dfo-ls}{conda\sphinxhyphen{}forge}:
\begin{quote}

\begin{sphinxVerbatim}[commandchars=\\\{\}]
\PYGZdl{}\PYG{+w}{ }conda\PYG{+w}{ }install\PYG{+w}{ }\PYGZhy{}c\PYG{+w}{ }conda\PYGZhy{}forge\PYG{+w}{ }dfo\PYGZhy{}ls
\end{sphinxVerbatim}
\end{quote}


\section{Installation using pip}
\label{\detokenize{install:installation-using-pip}}
\sphinxAtStartPar
For easy installation, use \sphinxstyleemphasis{pip} (\sphinxurl{http://www.pip-installer.org/}) as root:
\begin{quote}

\begin{sphinxVerbatim}[commandchars=\\\{\}]
\PYGZdl{}\PYG{+w}{ }pip\PYG{+w}{ }install\PYG{+w}{ }DFO\PYGZhy{}LS
\end{sphinxVerbatim}
\end{quote}

\sphinxAtStartPar
Note that if an older install of DFO\sphinxhyphen{}LS is present on your system you can use:
\begin{quote}

\begin{sphinxVerbatim}[commandchars=\\\{\}]
\PYGZdl{}\PYG{+w}{ }pip\PYG{+w}{ }install\PYG{+w}{ }\PYGZhy{}\PYGZhy{}upgrade\PYG{+w}{ }DFO\PYGZhy{}LS
\end{sphinxVerbatim}
\end{quote}

\sphinxAtStartPar
to upgrade DFO\sphinxhyphen{}LS to the latest version.


\section{Manual installation}
\label{\detokenize{install:manual-installation}}
\sphinxAtStartPar
Alternatively, you can download the source code from \sphinxhref{https://github.com/numericalalgorithmsgroup/dfols}{Github} and unpack as follows:
\begin{quote}

\begin{sphinxVerbatim}[commandchars=\\\{\}]
\PYGZdl{}\PYG{+w}{ }git\PYG{+w}{ }clone\PYG{+w}{ }https://github.com/numericalalgorithmsgroup/dfols
\PYGZdl{}\PYG{+w}{ }\PYG{n+nb}{cd}\PYG{+w}{ }dfols
\end{sphinxVerbatim}
\end{quote}

\sphinxAtStartPar
DFO\sphinxhyphen{}LS is written in pure Python and requires no compilation. It can be installed using:
\begin{quote}

\begin{sphinxVerbatim}[commandchars=\\\{\}]
\PYGZdl{}\PYG{+w}{ }pip\PYG{+w}{ }install\PYG{+w}{ }.
\end{sphinxVerbatim}
\end{quote}

\sphinxAtStartPar
To upgrade DFO\sphinxhyphen{}LS to the latest version, navigate to the top\sphinxhyphen{}level directory (i.e. the one containing \sphinxcode{\sphinxupquote{pyproject.toml}}) and rerun the installation using \sphinxcode{\sphinxupquote{pip}}, as above:
\begin{quote}

\begin{sphinxVerbatim}[commandchars=\\\{\}]
\PYGZdl{}\PYG{+w}{ }git\PYG{+w}{ }pull
\PYGZdl{}\PYG{+w}{ }pip\PYG{+w}{ }install\PYG{+w}{ }.
\end{sphinxVerbatim}
\end{quote}


\section{Testing}
\label{\detokenize{install:testing}}
\sphinxAtStartPar
If you installed DFO\sphinxhyphen{}LS manually, you can test your installation using the pytest package:
\begin{quote}

\begin{sphinxVerbatim}[commandchars=\\\{\}]
\PYGZdl{}\PYG{+w}{ }pip\PYG{+w}{ }install\PYG{+w}{ }pytest
\PYGZdl{}\PYG{+w}{ }python\PYG{+w}{ }\PYGZhy{}m\PYG{+w}{ }pytest\PYG{+w}{ }\PYGZhy{}\PYGZhy{}pyargs\PYG{+w}{ }dfols
\end{sphinxVerbatim}
\end{quote}

\sphinxAtStartPar
Alternatively, this documentation provides some simple examples of how to run DFO\sphinxhyphen{}LS.


\section{Uninstallation}
\label{\detokenize{install:uninstallation}}
\sphinxAtStartPar
If DFO\sphinxhyphen{}LS was installed using \sphinxstyleemphasis{pip} you can uninstall as follows:
\begin{quote}

\begin{sphinxVerbatim}[commandchars=\\\{\}]
\PYGZdl{}\PYG{+w}{ }pip\PYG{+w}{ }uninstall\PYG{+w}{ }DFO\PYGZhy{}LS
\end{sphinxVerbatim}
\end{quote}

\sphinxAtStartPar
If DFO\sphinxhyphen{}LS was installed manually you have to remove the installed files by hand (located in your python site\sphinxhyphen{}packages directory).

\sphinxstepscope


\chapter{Overview}
\label{\detokenize{info:overview}}\label{\detokenize{info::doc}}

\section{When to use DFO\sphinxhyphen{}LS}
\label{\detokenize{info:when-to-use-dfo-ls}}
\sphinxAtStartPar
DFO\sphinxhyphen{}LS is designed to solve the nonlinear least\sphinxhyphen{}squares minimization problem (with optional convex constraints).
\begin{equation*}
\begin{split}\min_{x\in\mathbb{R}^n}  &\quad  f(x) := \sum_{i=1}^{m}r_{i}(x)^2 + h(x) \\
\text{s.t.} &\quad  a \leq x \leq b\\
            &\quad x \in C := C_1 \cap \cdots \cap C_n, \quad \text{all $C_i$ convex}\end{split}
\end{equation*}
\sphinxAtStartPar
We call \(f(x)\) the objective function, \(r_i(x)\) the residual functions (or simply residuals), and \(h(x)\) the regularizer.
\(C\) is the intersection of multiple convex sets given as input by the user.

\sphinxAtStartPar
DFO\sphinxhyphen{}LS is a \sphinxstyleemphasis{derivative\sphinxhyphen{}free} optimization algorithm, which means it does not require the user to provide the derivatives of \(f(x)\) or \(r_i(x)\), nor does it attempt to estimate them internally (by using finite differencing, for instance).

\sphinxAtStartPar
There are two main situations when using a derivative\sphinxhyphen{}free algorithm (such as DFO\sphinxhyphen{}LS) is preferable to a derivative\sphinxhyphen{}based algorithm (which is the vast majority of least\sphinxhyphen{}squares solvers).

\sphinxAtStartPar
If \sphinxstylestrong{the residuals are noisy}, then calculating or even estimating their derivatives may be impossible (or at least very inaccurate). By noisy, we mean that if we evaluate \(r_i(x)\) multiple times at the same value of \(x\), we get different results. This may happen when a Monte Carlo simulation is used, for instance, or \(r_i(x)\) involves performing a physical experiment.

\sphinxAtStartPar
If \sphinxstylestrong{the residuals are expensive to evaluate}, then estimating derivatives (which requires \(n\) evaluations of each \(r_i(x)\) for every point of interest \(x\)) may be prohibitively expensive. Derivative\sphinxhyphen{}free methods are designed to solve the problem with the fewest number of evaluations of the objective as possible.

\sphinxAtStartPar
\sphinxstylestrong{However, if you have provide (or a solver can estimate) derivatives} of \(r_i(x)\), then it is probably a good idea to use one of the many derivative\sphinxhyphen{}based solvers (such as \sphinxhref{https://docs.scipy.org/doc/scipy/reference/generated/scipy.optimize.least\_squares.html}{one from the SciPy library}).


\section{Parameter Fitting}
\label{\detokenize{info:parameter-fitting}}
\sphinxAtStartPar
A very common problem in many quantitative disciplines is fitting parameters to observed data. Typically, this means that we have developed a model for some proccess, which takes a vector of (known) inputs \(\mathrm{obs}\in\mathbb{R}^N\) and some model parameters \(x=(x_1, \ldots, x_n)\in\mathbb{R}^n\), and computes a (predicted) quantity of interest \(y\in\mathbb{R}\):
\begin{equation*}
\begin{split}y = \mathrm{model}(\mathrm{obs}, x)\end{split}
\end{equation*}
\sphinxAtStartPar
For this model to be useful, we need to determine a suitable choice for the parameters \(x\), which typically cannot be directly observed. A common way of doing this is to calibrate from observed relationships.

\sphinxAtStartPar
Suppose we have some observations of the input\sphinxhyphen{}to\sphinxhyphen{}output relationship. That is, we have data
\begin{equation*}
\begin{split}(\mathrm{obs}_1, y_1), \ldots, (\mathrm{obs}_m, y_m)\end{split}
\end{equation*}
\sphinxAtStartPar
Then, we try to find the parameters \(x\) which produce the best possible fit to these observations by minimizing the sum\sphinxhyphen{}of\sphinxhyphen{}squares of the prediction errors:
\begin{equation*}
\begin{split}\min_{x\in\mathbb{R}^n}  \quad  f(x) := \sum_{i=1}^{m}(y_i - \mathrm{model}(\mathrm{obs}_i, x))^2\end{split}
\end{equation*}
\sphinxAtStartPar
which is in the least\sphinxhyphen{}squares form required by DFO\sphinxhyphen{}LS.

\sphinxAtStartPar
As described above, DFO\sphinxhyphen{}LS is a particularly good choice for parameter fitting when the model has noise (e.g. Monte Carlo simulation) or is expensive to evaluate.


\section{Solving Nonlinear Systems of Equations}
\label{\detokenize{info:solving-nonlinear-systems-of-equations}}
\sphinxAtStartPar
Suppose we wish to solve the system of nonlinear equations: find \(x\in\mathbb{R}^n\) satisfying
\begin{equation*}
\begin{split}r_1(x) &= 0 \\
r_2(x) &= 0 \\
&\vdots \\
r_m(x) &= 0\end{split}
\end{equation*}
\sphinxAtStartPar
Such problems can have no solutions, one solution, or many solutions (possibly infinitely many). Often, but certainly not always, the number of solutions depends on whether there are more equations or unknowns: if \(m<n\) we say the system is underdetermined (and there are often multiple solutions), if \(m=n\) we say the system is square (and there is often only one solution), and if \(m>n\) we say the system is overdetermined (and there are often no solutions).

\sphinxAtStartPar
This is not always true \textendash{} there is no solution to the underdetermined system when \(m=1\) and \(n=2\) and we choose \(r_1(x)=\sin(x_1+x_2)-2\), for example.
Similarly, if we take \(n=1\) and \(r_i(x)=i (x-1)(x-2)\), we can make \(m\) as large as we like while keeping \(x=1\) and \(x=2\) as solutions (to the overdetermined system).

\sphinxAtStartPar
If no solution exists, it makes sense to instead search for an \(x\) which approximately satisfies each equation. A common way to do this is to minimize the sum\sphinxhyphen{}of\sphinxhyphen{}squares of the left\sphinxhyphen{}hand\sphinxhyphen{}sides:
\begin{equation*}
\begin{split}\min_{x\in\mathbb{R}^n}  \quad  f(x) := \sum_{i=1}^{m}r_i(x)^2\end{split}
\end{equation*}
\sphinxAtStartPar
which is the form required by DFO\sphinxhyphen{}LS.

\sphinxAtStartPar
If a solution does exist, then this formulation will also find this (where we will get \(f=0\) at the solution).

\sphinxAtStartPar
\sphinxstylestrong{Which solution?} DFO\sphinxhyphen{}LS, and most similar software, will only find one solution to a set of nonlinear equations. Which one it finds is very difficult to predict, and depends very strongly on the point where the solver is started from. Often it finds the closest solution, but there are no guarantees this will be the case. If you need to find all/multiple solutions for your problem, consider techniques such as \sphinxhref{http://www.sciencedirect.com/science/article/pii/0022247X83900550}{deflation}.


\section{Details of the DFO\sphinxhyphen{}LS Algorithm}
\label{\detokenize{info:details-of-the-dfo-ls-algorithm}}
\sphinxAtStartPar
DFO\sphinxhyphen{}LS is a type of \sphinxstyleemphasis{trust\sphinxhyphen{}region} method, a common category of optimization algorithms for nonconvex problems. Given a current estimate of the solution \(x_k\), we compute a model which approximates the objective \(m_k(s)\approx f(x_k+s)\) (for small steps \(s\)), and maintain a value \(\Delta_k>0\) (called the \sphinxstyleemphasis{trust region radius}) which measures the size of \(s\) for which the approximation is good.

\sphinxAtStartPar
At each step, we compute a trial step \(s_k\) designed to make our approximation \(m_k(s)\) small (this task is called the \sphinxstyleemphasis{trust region subproblem}). We evaluate the objective at this new point, and if this provided a good decrease in the objective, we take the step (\(x_{k+1}=x_k+s_k\)), otherwise we stay put (\(x_{k+1}=x_k\)). Based on this information, we choose a new value \(\Delta_{k+1}\), and repeat the process.

\sphinxAtStartPar
In DFO\sphinxhyphen{}LS, we construct our approximation \(m_k(s)\) by interpolating a linear approximation for each residual \(r_i(x)\) at several points close to \(x_k\). To make sure our interpolated model is accurate, we need to regularly check that the points are well\sphinxhyphen{}spaced, and move them if they aren’t (i.e. improve the geometry of our interpolation points).

\sphinxAtStartPar
A complete description of the DFO\sphinxhyphen{}LS algorithm is given in our papers \sphinxcite{userguide:cfmr2018}, \sphinxcite{userguide:hr2022} and \sphinxcite{userguide:llr2024}.


\section{References}
\label{\detokenize{info:references}}
\sphinxstepscope


\chapter{Using DFO\sphinxhyphen{}LS}
\label{\detokenize{userguide:using-dfo-ls}}\label{\detokenize{userguide::doc}}
\sphinxAtStartPar
This section describes the main interface to DFO\sphinxhyphen{}LS and how to use it.


\section{Nonlinear Least\sphinxhyphen{}Squares Minimization}
\label{\detokenize{userguide:nonlinear-least-squares-minimization}}
\sphinxAtStartPar
DFO\sphinxhyphen{}LS is designed to solve the local optimization problem
\begin{equation*}
\begin{split}\min_{x\in\mathbb{R}^n}  &\quad  f(x) := \sum_{i=1}^{m}r_{i}(x)^2 + h(x) \\
\text{s.t.} &\quad  a \leq x \leq b\\
            &\quad x \in C := C_1 \cap \cdots \cap C_n, \quad \text{all $C_i$ convex}\end{split}
\end{equation*}
\sphinxAtStartPar
where the bound constraints \(a \leq x \leq b\) and general convex constraints \(x\in C\) are optional. The upper and lower bounds on the variables are non\sphinxhyphen{}relaxable (i.e. DFO\sphinxhyphen{}LS will never ask to evaluate a point outside the bounds). The general convex constraints are also non\sphinxhyphen{}relaxable, but they may be slightly violated at some points from rounding errors.
The function \(h(x)\) is an optional regularizer, often used to avoid overfitting, and must be Lipschitz continuous and convex (but possibly non\sphinxhyphen{}differentiable).
A common choice is \(h(x)=\lambda \|x\|_1\) (called L1 regularization or LASSO) for \(\lambda>0\).
Note that in the case of Tikhonov regularization/ridge regression, \(h(x)=\lambda\|x\|_2^2\) is not Lipschitz continuous, so should instead be incorporated by adding an extra term into the least\sphinxhyphen{}squares sum, \(r_{m+1}(x)=\sqrt{\lambda} \|x\|_2\).

\sphinxAtStartPar
DFO\sphinxhyphen{}LS iteratively constructs an interpolation\sphinxhyphen{}based model for the objective, and determines a step using a trust\sphinxhyphen{}region framework.
For an in\sphinxhyphen{}depth technical description of the algorithm see the papers \sphinxcite{userguide:cfmr2018}, \sphinxcite{userguide:hr2022} and \sphinxcite{userguide:llr2024}.


\section{How to use DFO\sphinxhyphen{}LS}
\label{\detokenize{userguide:how-to-use-dfo-ls}}
\sphinxAtStartPar
The main interface to DFO\sphinxhyphen{}LS is via the function \sphinxcode{\sphinxupquote{solve}}
\begin{quote}

\begin{sphinxVerbatim}[commandchars=\\\{\}]
\PYG{n}{soln} \PYG{o}{=} \PYG{n}{dfols}\PYG{o}{.}\PYG{n}{solve}\PYG{p}{(}\PYG{n}{objfun}\PYG{p}{,} \PYG{n}{x0}\PYG{p}{)}
\end{sphinxVerbatim}
\end{quote}

\sphinxAtStartPar
The input \sphinxcode{\sphinxupquote{objfun}} is a Python function which takes an input \(x\in\mathbb{R}^n\) and returns the vector of residuals \([r_1(x)\: \cdots \: r_m(x)]\in\mathbb{R}^m\). Both the input and output of \sphinxcode{\sphinxupquote{objfun}} must be one\sphinxhyphen{}dimensional NumPy arrays (i.e. with \sphinxcode{\sphinxupquote{x.shape == (n,)}} and \sphinxcode{\sphinxupquote{objfun(x).shape == (m,)}}).

\sphinxAtStartPar
The input \sphinxcode{\sphinxupquote{x0}} is the starting point for the solver, and (where possible) should be set to be the best available estimate of the true solution \(x_{min}\in\mathbb{R}^n\). It should be specified as a one\sphinxhyphen{}dimensional NumPy array (i.e. with \sphinxcode{\sphinxupquote{x0.shape == (n,)}}).
As DFO\sphinxhyphen{}LS is a local solver, providing different values for \sphinxcode{\sphinxupquote{x0}} may cause it to return different solutions, with possibly different objective values.

\sphinxAtStartPar
The output of \sphinxcode{\sphinxupquote{dfols.solve}} is an object containing:
\begin{itemize}
\item {} 
\sphinxAtStartPar
\sphinxcode{\sphinxupquote{soln.x}} \sphinxhyphen{} an estimate of the solution, \(x_{min}\in\mathbb{R}^n\), a one\sphinxhyphen{}dimensional NumPy array.

\item {} 
\sphinxAtStartPar
\sphinxcode{\sphinxupquote{soln.resid}} \sphinxhyphen{} the vector of residuals at the calculated solution, \([r_1(x_{min})\:\cdots\: r_m(x_{min})]\), a one\sphinxhyphen{}dimensional NumPy array.

\item {} 
\sphinxAtStartPar
\sphinxcode{\sphinxupquote{soln.f}} \sphinxhyphen{} the objective value at the calculated solution, \(f(x_{min})\), a Float.

\item {} 
\sphinxAtStartPar
\sphinxcode{\sphinxupquote{soln.jacobian}} \sphinxhyphen{} an estimate of the Jacobian matrix of first derivatives of the residuals, \(J_{i,j} \approx \partial r_i(x_{min})/\partial x_j\), a NumPy array of size \(m\times n\).

\item {} 
\sphinxAtStartPar
\sphinxcode{\sphinxupquote{soln.nf}} \sphinxhyphen{} the number of evaluations of \sphinxcode{\sphinxupquote{objfun}} that the algorithm needed, an Integer.

\item {} 
\sphinxAtStartPar
\sphinxcode{\sphinxupquote{soln.nx}} \sphinxhyphen{} the number of points \(x\) at which \sphinxcode{\sphinxupquote{objfun}} was evaluated, an Integer. This may be different to \sphinxcode{\sphinxupquote{soln.nf}} if sample averaging is used.

\item {} 
\sphinxAtStartPar
\sphinxcode{\sphinxupquote{soln.nruns}} \sphinxhyphen{} the number of runs performed by DFO\sphinxhyphen{}LS (more than 1 if using multiple restarts), an Integer.

\item {} 
\sphinxAtStartPar
\sphinxcode{\sphinxupquote{soln.flag}} \sphinxhyphen{} an exit flag, which can take one of several values (listed below), an Integer.

\item {} 
\sphinxAtStartPar
\sphinxcode{\sphinxupquote{soln.msg}} \sphinxhyphen{} a description of why the algorithm finished, a String.

\item {} 
\sphinxAtStartPar
\sphinxcode{\sphinxupquote{soln.diagnostic\_info}} \sphinxhyphen{} a table of diagnostic information showing the progress of the solver, a Pandas DataFrame.

\item {} 
\sphinxAtStartPar
\sphinxcode{\sphinxupquote{soln.xmin\_eval\_num}} \sphinxhyphen{} an integer representing which evaluation point (i.e. same as \sphinxcode{\sphinxupquote{soln.nx}}) gave the solution \sphinxcode{\sphinxupquote{soln.x}}. Evaluation counts are 1\sphinxhyphen{}indexed, to match the logging information.

\item {} 
\sphinxAtStartPar
\sphinxcode{\sphinxupquote{soln.jacmin\_eval\_nums}} \sphinxhyphen{} a NumPy integer array of length \sphinxcode{\sphinxupquote{npt}} with the evaluation point numbers (i.e. same as \sphinxcode{\sphinxupquote{soln.nx}}) used to build \sphinxcode{\sphinxupquote{soln.jacobian}} via linear interpolation to the residual values at these points. Evaluation counts are 1\sphinxhyphen{}indexed, to match the logging information. This array will usually, but not always, include \sphinxcode{\sphinxupquote{soln.xmin\_eval\_num}}.

\end{itemize}

\sphinxAtStartPar
The possible values of \sphinxcode{\sphinxupquote{soln.flag}} are defined by the following variables:
\begin{itemize}
\item {} 
\sphinxAtStartPar
\sphinxcode{\sphinxupquote{soln.EXIT\_SUCCESS}} \sphinxhyphen{} DFO\sphinxhyphen{}LS terminated successfully (the objective value or trust region radius are sufficiently small).

\item {} 
\sphinxAtStartPar
\sphinxcode{\sphinxupquote{soln.EXIT\_MAXFUN\_WARNING}} \sphinxhyphen{} maximum allowed objective evaluations reached. This is the most likely return value when using multiple restarts.

\item {} 
\sphinxAtStartPar
\sphinxcode{\sphinxupquote{soln.EXIT\_SLOW\_WARNING}} \sphinxhyphen{} maximum number of slow iterations reached.

\item {} 
\sphinxAtStartPar
\sphinxcode{\sphinxupquote{soln.EXIT\_FALSE\_SUCCESS\_WARNING}} \sphinxhyphen{} DFO\sphinxhyphen{}LS reached the maximum number of restarts which decreased the objective, but to a worse value than was found in a previous run.

\item {} 
\sphinxAtStartPar
\sphinxcode{\sphinxupquote{soln.EXIT\_TR\_INCREASE\_WARNING}} \sphinxhyphen{} model increase when solving the trust region subproblem with multiple arbitrary constraints.

\item {} 
\sphinxAtStartPar
\sphinxcode{\sphinxupquote{soln.EXIT\_INPUT\_ERROR}} \sphinxhyphen{} error in the inputs.

\item {} 
\sphinxAtStartPar
\sphinxcode{\sphinxupquote{soln.EXIT\_TR\_INCREASE\_ERROR}} \sphinxhyphen{} error occurred when solving the trust region subproblem.

\item {} 
\sphinxAtStartPar
\sphinxcode{\sphinxupquote{soln.EXIT\_LINALG\_ERROR}} \sphinxhyphen{} linear algebra error, e.g. the interpolation points produced a singular linear system.

\item {} 
\sphinxAtStartPar
\sphinxcode{\sphinxupquote{soln.EXIT\_EVAL\_ERROR}} \sphinxhyphen{} the objective function returned a NaN value when evaluating at a new trial point.

\end{itemize}

\sphinxAtStartPar
These variables are defined in the \sphinxcode{\sphinxupquote{soln}} object, so can be accessed with, for example
\begin{quote}

\begin{sphinxVerbatim}[commandchars=\\\{\}]
\PYG{k}{if} \PYG{n}{soln}\PYG{o}{.}\PYG{n}{flag} \PYG{o}{==} \PYG{n}{soln}\PYG{o}{.}\PYG{n}{EXIT\PYGZus{}SUCCESS}\PYG{p}{:}
    \PYG{n+nb}{print}\PYG{p}{(}\PYG{l+s+s2}{\PYGZdq{}}\PYG{l+s+s2}{Success!}\PYG{l+s+s2}{\PYGZdq{}}\PYG{p}{)}
\end{sphinxVerbatim}
\end{quote}


\section{Optional Arguments}
\label{\detokenize{userguide:optional-arguments}}
\sphinxAtStartPar
The \sphinxcode{\sphinxupquote{solve}} function has several optional arguments which the user may provide:
\begin{quote}

\begin{sphinxVerbatim}[commandchars=\\\{\}]
\PYG{n}{dfols}\PYG{o}{.}\PYG{n}{solve}\PYG{p}{(}\PYG{n}{objfun}\PYG{p}{,} \PYG{n}{x0}\PYG{p}{,}
            \PYG{n}{h}\PYG{o}{=}\PYG{k+kc}{None}\PYG{p}{,} \PYG{n}{lh}\PYG{o}{=}\PYG{k+kc}{None}\PYG{p}{,} \PYG{n}{prox\PYGZus{}uh}\PYG{o}{=}\PYG{k+kc}{None}\PYG{p}{,}
            \PYG{n}{argsf}\PYG{o}{=}\PYG{p}{(}\PYG{p}{)}\PYG{p}{,} \PYG{n}{argsh}\PYG{o}{=}\PYG{p}{(}\PYG{p}{)}\PYG{p}{,} \PYG{n}{argsprox}\PYG{o}{=}\PYG{p}{(}\PYG{p}{)}\PYG{p}{,}
            \PYG{n}{bounds}\PYG{o}{=}\PYG{k+kc}{None}\PYG{p}{,} \PYG{n}{projections}\PYG{o}{=}\PYG{p}{[}\PYG{p}{]}\PYG{p}{,} \PYG{n}{npt}\PYG{o}{=}\PYG{k+kc}{None}\PYG{p}{,} \PYG{n}{rhobeg}\PYG{o}{=}\PYG{k+kc}{None}\PYG{p}{,}
            \PYG{n}{rhoend}\PYG{o}{=}\PYG{l+m+mf}{1e\PYGZhy{}8}\PYG{p}{,} \PYG{n}{maxfun}\PYG{o}{=}\PYG{k+kc}{None}\PYG{p}{,} \PYG{n}{nsamples}\PYG{o}{=}\PYG{k+kc}{None}\PYG{p}{,}
            \PYG{n}{user\PYGZus{}params}\PYG{o}{=}\PYG{k+kc}{None}\PYG{p}{,} \PYG{n}{objfun\PYGZus{}has\PYGZus{}noise}\PYG{o}{=}\PYG{k+kc}{False}\PYG{p}{,}
            \PYG{n}{scaling\PYGZus{}within\PYGZus{}bounds}\PYG{o}{=}\PYG{k+kc}{False}\PYG{p}{,}
            \PYG{n}{do\PYGZus{}logging}\PYG{o}{=}\PYG{k+kc}{True}\PYG{p}{,} \PYG{n}{print\PYGZus{}progress}\PYG{o}{=}\PYG{k+kc}{False}\PYG{p}{)}
\end{sphinxVerbatim}
\end{quote}

\sphinxAtStartPar
These arguments are:
\begin{itemize}
\item {} 
\sphinxAtStartPar
\sphinxcode{\sphinxupquote{h}} \sphinxhyphen{} the regularizer function which takes an input \(x\in\mathbb{R}^n\) and returns \(h(x)\).

\item {} 
\sphinxAtStartPar
\sphinxcode{\sphinxupquote{lh}} \sphinxhyphen{} the \sphinxhref{https://en.wikipedia.org/wiki/Lipschitz\_continuity}{Lipschitz constant} (with respect to the Euclidean norm on \(\mathbb{R}^n\)) of \(h(x)\), a positive number if \sphinxcode{\sphinxupquote{h}} given. For example, if \(h(x)=\lambda \|x\|_1\) for \(\lambda>0\), then \(L_h=\lambda \sqrt{n}\).

\item {} 
\sphinxAtStartPar
\sphinxcode{\sphinxupquote{prox\_uh}} \sphinxhyphen{} the \sphinxhref{https://en.wikipedia.org/wiki/Proximal\_operator}{proximal operator} of \(h(x)\). This function has the form \sphinxcode{\sphinxupquote{prox\_uh(x, u)}}, where \(x\in \mathbb{R}^n\) and \(u>0\), and returns \(\operatorname{prox}_{uh}(x)\). For example, if \(h(x)=\lambda \|x\|_1\) for \(\lambda>0\), then \sphinxcode{\sphinxupquote{prox\_uh(x, u) = np.sign(x) * np.maximum(np.abs(x) \sphinxhyphen{} lambda*u, 0)}}. More examples of proximal operators may be found on \sphinxhref{https://proximity-operator.net/}{this page}.

\item {} 
\sphinxAtStartPar
\sphinxcode{\sphinxupquote{argsf}} \sphinxhyphen{} a tuple of extra arguments passed to the objective function \sphinxcode{\sphinxupquote{objfun(x, *argsf)}}.

\item {} 
\sphinxAtStartPar
\sphinxcode{\sphinxupquote{argsh}} \sphinxhyphen{} a tuple of extra arguments passed to the regularizer \sphinxcode{\sphinxupquote{h(x, *argsh)}}.

\item {} 
\sphinxAtStartPar
\sphinxcode{\sphinxupquote{argsprox}} \sphinxhyphen{} a tuple of extra arguments passed to the proximal operator \sphinxcode{\sphinxupquote{prox\_uh(x, u, *argsprox)}}.

\item {} 
\sphinxAtStartPar
\sphinxcode{\sphinxupquote{bounds}} \sphinxhyphen{} a tuple \sphinxcode{\sphinxupquote{(lower, upper)}} with the vectors \(a\) and \(b\) of lower and upper bounds on \(x\) (default is \(a_i=-10^{20}\) and \(b_i=10^{20}\)). To set bounds for either \sphinxcode{\sphinxupquote{lower}} or \sphinxcode{\sphinxupquote{upper}}, but not both, pass a tuple \sphinxcode{\sphinxupquote{(lower, None)}} or \sphinxcode{\sphinxupquote{(None, upper)}}.

\item {} 
\sphinxAtStartPar
\sphinxcode{\sphinxupquote{projections}} \sphinxhyphen{} a list \sphinxcode{\sphinxupquote{{[}f1,f2,...,fn{]}}} of functions that each take as input a point \sphinxcode{\sphinxupquote{x}} and return a new point \sphinxcode{\sphinxupquote{y}}. The new point \sphinxcode{\sphinxupquote{y}} should be given by the projection of \sphinxcode{\sphinxupquote{x}} onto a closed convex set. The intersection of all sets corresponding to a function must be non\sphinxhyphen{}empty.

\item {} 
\sphinxAtStartPar
\sphinxcode{\sphinxupquote{npt}} \sphinxhyphen{} the number of interpolation points to use (default is \sphinxcode{\sphinxupquote{len(x0)+1}}). If using restarts, this is the number of points to use in the first run of the solver, before any restarts (and may be optionally increased via settings in \sphinxcode{\sphinxupquote{user\_params}}).

\item {} 
\sphinxAtStartPar
\sphinxcode{\sphinxupquote{rhobeg}} \sphinxhyphen{} the initial value of the trust region radius (default is \(0.1\max(\|x_0\|_{\infty}, 1)\), or 0.1 if \sphinxcode{\sphinxupquote{scaling\_within\_bounds}}).

\item {} 
\sphinxAtStartPar
\sphinxcode{\sphinxupquote{rhoend}} \sphinxhyphen{} minimum allowed value of trust region radius, which determines when a successful termination occurs (default is \(10^{-8}\)).

\item {} 
\sphinxAtStartPar
\sphinxcode{\sphinxupquote{maxfun}} \sphinxhyphen{} the maximum number of objective evaluations the algorithm may request (default is \(\min(100(n+1),1000)\)).

\item {} 
\sphinxAtStartPar
\sphinxcode{\sphinxupquote{nsamples}} \sphinxhyphen{} a Python function \sphinxcode{\sphinxupquote{nsamples(delta, rho, iter, nrestarts)}} which returns the number of times to evaluate \sphinxcode{\sphinxupquote{objfun}} at a given point. This is only applicable for objectives with stochastic noise, when averaging multiple evaluations at the same point produces a more accurate value. The input parameters are the trust region radius (\sphinxcode{\sphinxupquote{delta}}), the lower bound on the trust region radius (\sphinxcode{\sphinxupquote{rho}}), how many iterations the algorithm has been running for (\sphinxcode{\sphinxupquote{iter}}), and how many restarts have been performed (\sphinxcode{\sphinxupquote{nrestarts}}). Default is no averaging (i.e. \sphinxcode{\sphinxupquote{nsamples(delta, rho, iter, nrestarts)=1}}).

\item {} 
\sphinxAtStartPar
\sphinxcode{\sphinxupquote{user\_params}} \sphinxhyphen{} a Python dictionary \sphinxcode{\sphinxupquote{\{\textquotesingle{}param1\textquotesingle{}: val1, \textquotesingle{}param2\textquotesingle{}:val2, ...\}}} of optional parameters. A full list of available options is given in the next section {\hyperref[\detokenize{advanced::doc}]{\sphinxcrossref{\DUrole{doc}{Advanced Usage}}}}.

\item {} 
\sphinxAtStartPar
\sphinxcode{\sphinxupquote{objfun\_has\_noise}} \sphinxhyphen{} a flag to indicate whether or not \sphinxcode{\sphinxupquote{objfun}} has stochastic noise; i.e. will calling \sphinxcode{\sphinxupquote{objfun(x)}} multiple times at the same value of \sphinxcode{\sphinxupquote{x}} give different results? This is used to set some sensible default parameters (including using multiple restarts), all of which can be overridden by the values provided in \sphinxcode{\sphinxupquote{user\_params}}.

\item {} 
\sphinxAtStartPar
\sphinxcode{\sphinxupquote{scaling\_within\_bounds}} \sphinxhyphen{} a flag to indicate whether the algorithm should internally shift and scale the entries of \sphinxcode{\sphinxupquote{x}} so that the bounds become \(0 \leq x \leq 1\). This is useful is you are setting \sphinxcode{\sphinxupquote{bounds}} and the bounds have different orders of magnitude. If \sphinxcode{\sphinxupquote{scaling\_within\_bounds=True}}, the values of \sphinxcode{\sphinxupquote{rhobeg}} and \sphinxcode{\sphinxupquote{rhoend}} apply to the \sphinxstyleemphasis{shifted} variables.

\item {} 
\sphinxAtStartPar
\sphinxcode{\sphinxupquote{do\_logging}} \sphinxhyphen{} a flag to indicate whether logging output should be produced. This is not automatically visible unless you use the Python \sphinxhref{https://docs.python.org/3/library/logging.html}{logging} module (see below for simple usage).

\item {} 
\sphinxAtStartPar
\sphinxcode{\sphinxupquote{print\_progress}} \sphinxhyphen{} a flag to indicate whether to print a per\sphinxhyphen{}iteration progress log to terminal.

\end{itemize}

\sphinxAtStartPar
In general when using optimization software, it is good practice to scale your variables so that moving each by a given amount has approximately the same impact on the objective function.
The \sphinxcode{\sphinxupquote{scaling\_within\_bounds}} flag is designed to provide an easy way to achieve this, if you have set the bounds \sphinxcode{\sphinxupquote{lower}} and \sphinxcode{\sphinxupquote{upper}}.


\section{A Simple Example}
\label{\detokenize{userguide:a-simple-example}}
\sphinxAtStartPar
Suppose we wish to minimize the \sphinxhref{https://en.wikipedia.org/wiki/Rosenbrock\_function}{Rosenbrock test function}:
\begin{equation*}
\begin{split}\min_{(x_1,x_2)\in\mathbb{R}^2}  &\quad  100(x_2-x_1^2)^2 + (1-x_1)^2 \\\end{split}
\end{equation*}
\sphinxAtStartPar
This function has exactly one local minimum \(f(x_{min})=0\) at \(x_{min}=(1,1)\). We can write this as a least\sphinxhyphen{}squares problem as:
\begin{equation*}
\begin{split}\min_{(x_1,x_2)\in\mathbb{R}^2}  &\quad  [10(x_2-x_1^2)]^2 + [1-x_1]^2 \\\end{split}
\end{equation*}
\sphinxAtStartPar
A commonly\sphinxhyphen{}used starting point for testing purposes is \(x_0=(-1.2,1)\). The following script shows how to solve this problem using DFO\sphinxhyphen{}LS:
\begin{quote}

\begin{sphinxVerbatim}[commandchars=\\\{\}]
\PYG{c+c1}{\PYGZsh{} DFO\PYGZhy{}LS example: minimize the Rosenbrock function}
\PYG{k+kn}{from} \PYG{n+nn}{\PYGZus{}\PYGZus{}future\PYGZus{}\PYGZus{}} \PYG{k+kn}{import} \PYG{n}{print\PYGZus{}function}
\PYG{k+kn}{import} \PYG{n+nn}{numpy} \PYG{k}{as} \PYG{n+nn}{np}
\PYG{k+kn}{import} \PYG{n+nn}{dfols}

\PYG{c+c1}{\PYGZsh{} Define the objective function}
\PYG{k}{def} \PYG{n+nf}{rosenbrock}\PYG{p}{(}\PYG{n}{x}\PYG{p}{)}\PYG{p}{:}
    \PYG{k}{return} \PYG{n}{np}\PYG{o}{.}\PYG{n}{array}\PYG{p}{(}\PYG{p}{[}\PYG{l+m+mf}{10.0} \PYG{o}{*} \PYG{p}{(}\PYG{n}{x}\PYG{p}{[}\PYG{l+m+mi}{1}\PYG{p}{]} \PYG{o}{\PYGZhy{}} \PYG{n}{x}\PYG{p}{[}\PYG{l+m+mi}{0}\PYG{p}{]} \PYG{o}{*}\PYG{o}{*} \PYG{l+m+mi}{2}\PYG{p}{)}\PYG{p}{,} \PYG{l+m+mf}{1.0} \PYG{o}{\PYGZhy{}} \PYG{n}{x}\PYG{p}{[}\PYG{l+m+mi}{0}\PYG{p}{]}\PYG{p}{]}\PYG{p}{)}

\PYG{c+c1}{\PYGZsh{} Define the starting point}
\PYG{n}{x0} \PYG{o}{=} \PYG{n}{np}\PYG{o}{.}\PYG{n}{array}\PYG{p}{(}\PYG{p}{[}\PYG{o}{\PYGZhy{}}\PYG{l+m+mf}{1.2}\PYG{p}{,} \PYG{l+m+mf}{1.0}\PYG{p}{]}\PYG{p}{)}

\PYG{c+c1}{\PYGZsh{} Call DFO\PYGZhy{}LS}
\PYG{n}{soln} \PYG{o}{=} \PYG{n}{dfols}\PYG{o}{.}\PYG{n}{solve}\PYG{p}{(}\PYG{n}{rosenbrock}\PYG{p}{,} \PYG{n}{x0}\PYG{p}{)}

\PYG{c+c1}{\PYGZsh{} Display output}
\PYG{n+nb}{print}\PYG{p}{(}\PYG{n}{soln}\PYG{p}{)}
\end{sphinxVerbatim}
\end{quote}

\sphinxAtStartPar
Note that DFO\sphinxhyphen{}LS is a randomized algorithm: in its first phase, it builds an internal approximation to the objective function by sampling it along random directions. In the code above, we set NumPy’s random seed for reproducibility over multiple runs, but this is not required. The output of this script, showing that DFO\sphinxhyphen{}LS finds the correct solution, is
\begin{quote}

\begin{sphinxVerbatim}[commandchars=\\\{\}]
****** DFO\PYGZhy{}LS Results ******
Solution xmin = [1. 1.]
Residual vector = [0. 0.]
Objective value f(xmin) = 0
Needed 33 objective evaluations (at 33 points)
Approximate Jacobian = [[\PYGZhy{}1.9982000e+01  1.0000000e+01]
 [\PYGZhy{}1.0000000e+00  1.0079924e\PYGZhy{}14]]
Solution xmin was evaluation point 33
Approximate Jacobian formed using evaluation points [29 31 32]
Exit flag = 0
Success: Objective is sufficiently small
****************************
\end{sphinxVerbatim}
\end{quote}

\sphinxAtStartPar
This and all following problems can be found in the \sphinxhref{https://github.com/numericalalgorithmsgroup/dfols/tree/master/examples}{examples} directory on the DFO\sphinxhyphen{}LS Github page.


\section{Adding Bounds and More Output}
\label{\detokenize{userguide:adding-bounds-and-more-output}}
\sphinxAtStartPar
We can extend the above script to add constraints. To add bound constraints alone, we can add the lines
\begin{quote}

\begin{sphinxVerbatim}[commandchars=\\\{\}]
\PYG{c+c1}{\PYGZsh{} Define bound constraints (lower \PYGZlt{}= x \PYGZlt{}= upper)}
\PYG{n}{lower} \PYG{o}{=} \PYG{n}{np}\PYG{o}{.}\PYG{n}{array}\PYG{p}{(}\PYG{p}{[}\PYG{o}{\PYGZhy{}}\PYG{l+m+mf}{10.0}\PYG{p}{,} \PYG{o}{\PYGZhy{}}\PYG{l+m+mf}{10.0}\PYG{p}{]}\PYG{p}{)}
\PYG{n}{upper} \PYG{o}{=} \PYG{n}{np}\PYG{o}{.}\PYG{n}{array}\PYG{p}{(}\PYG{p}{[}\PYG{l+m+mf}{0.9}\PYG{p}{,} \PYG{l+m+mf}{0.85}\PYG{p}{]}\PYG{p}{)}

\PYG{c+c1}{\PYGZsh{} Call DFO\PYGZhy{}LS (with bounds)}
\PYG{n}{soln} \PYG{o}{=} \PYG{n}{dfols}\PYG{o}{.}\PYG{n}{solve}\PYG{p}{(}\PYG{n}{rosenbrock}\PYG{p}{,} \PYG{n}{x0}\PYG{p}{,} \PYG{n}{bounds}\PYG{o}{=}\PYG{p}{(}\PYG{n}{lower}\PYG{p}{,} \PYG{n}{upper}\PYG{p}{)}\PYG{p}{)}
\end{sphinxVerbatim}
\end{quote}

\sphinxAtStartPar
DFO\sphinxhyphen{}LS correctly finds the solution to the constrained problem:
\begin{quote}

\begin{sphinxVerbatim}[commandchars=\\\{\}]
****** DFO\PYGZhy{}LS Results ******
Solution xmin = [0.9  0.81]
Residual vector = [0.  0.1]
Objective value f(xmin) = 0.01
Needed 56 objective evaluations (at 56 points)
Approximate Jacobian = [[\PYGZhy{}1.79999999e+01  1.00000000e+01]
 [\PYGZhy{}1.00000000e+00 \PYGZhy{}5.15519307e\PYGZhy{}10]]
Solution xmin was evaluation point 42
Approximate Jacobian formed using evaluation points [55 42 54]
Exit flag = 0
Success: rho has reached rhoend
****************************
\end{sphinxVerbatim}
\end{quote}

\sphinxAtStartPar
However, we also get a warning that our starting point was outside of the bounds:
\begin{quote}

\begin{sphinxVerbatim}[commandchars=\\\{\}]
RuntimeWarning: x0 above upper bound, adjusting
\end{sphinxVerbatim}
\end{quote}

\sphinxAtStartPar
DFO\sphinxhyphen{}LS automatically fixes this, and moves \(x_0\) to a point within the bounds, in this case \(x_0=(-1.2,0.85)\).

\sphinxAtStartPar
We can also get DFO\sphinxhyphen{}LS to print out more detailed information about its progress using the \sphinxhref{https://docs.python.org/3/library/logging.html}{logging} module. To do this, we need to add the following lines:
\begin{quote}

\begin{sphinxVerbatim}[commandchars=\\\{\}]
\PYG{k+kn}{import} \PYG{n+nn}{logging}
\PYG{n}{logging}\PYG{o}{.}\PYG{n}{basicConfig}\PYG{p}{(}\PYG{n}{level}\PYG{o}{=}\PYG{n}{logging}\PYG{o}{.}\PYG{n}{INFO}\PYG{p}{,} \PYG{n+nb}{format}\PYG{o}{=}\PYG{l+s+s1}{\PYGZsq{}}\PYG{l+s+si}{\PYGZpc{}(message)s}\PYG{l+s+s1}{\PYGZsq{}}\PYG{p}{)}

\PYG{c+c1}{\PYGZsh{} ... (call dfols.solve)}
\end{sphinxVerbatim}
\end{quote}

\sphinxAtStartPar
And for the simple bounds example we can now see each evaluation of \sphinxcode{\sphinxupquote{objfun}}:
\begin{quote}

\begin{sphinxVerbatim}[commandchars=\\\{\}]
Function eval 1 at point 1 has f = 39.65 at x = [\PYGZhy{}1.2   0.85]
Initialising (coordinate directions)
Function eval 2 at point 2 has f = 14.337296 at x = [\PYGZhy{}1.08  0.85]
Function eval 3 at point 3 has f = 55.25 at x = [\PYGZhy{}1.2   0.73]
...
Function eval 55 at point 55 has obj = 0.0100000000000225 at x = [0.9        0.80999998]
Function eval 56 at point 56 has obj = 0.01 at x = [0.9  0.81]
Did a total of 1 run(s)
\end{sphinxVerbatim}
\end{quote}

\sphinxAtStartPar
If we wanted to save this output to a file, we could replace the above call to \sphinxcode{\sphinxupquote{logging.basicConfig()}} with
\begin{quote}

\begin{sphinxVerbatim}[commandchars=\\\{\}]
\PYG{n}{logging}\PYG{o}{.}\PYG{n}{basicConfig}\PYG{p}{(}\PYG{n}{filename}\PYG{o}{=}\PYG{l+s+s2}{\PYGZdq{}}\PYG{l+s+s2}{myfile.log}\PYG{l+s+s2}{\PYGZdq{}}\PYG{p}{,} \PYG{n}{level}\PYG{o}{=}\PYG{n}{logging}\PYG{o}{.}\PYG{n}{INFO}\PYG{p}{,}
                    \PYG{n+nb}{format}\PYG{o}{=}\PYG{l+s+s1}{\PYGZsq{}}\PYG{l+s+si}{\PYGZpc{}(message)s}\PYG{l+s+s1}{\PYGZsq{}}\PYG{p}{,} \PYG{n}{filemode}\PYG{o}{=}\PYG{l+s+s1}{\PYGZsq{}}\PYG{l+s+s1}{w}\PYG{l+s+s1}{\PYGZsq{}}\PYG{p}{)}
\end{sphinxVerbatim}
\end{quote}

\sphinxAtStartPar
If you have logging for some parts of your code and you want to deactivate all DFO\sphinxhyphen{}LS logging, you can use the optional argument \sphinxcode{\sphinxupquote{do\_logging=False}} in \sphinxcode{\sphinxupquote{dfols.solve()}}.

\sphinxAtStartPar
An alternative option available is to get DFO\sphinxhyphen{}LS to print to terminal progress information every iteration, by setting the optional argument \sphinxcode{\sphinxupquote{print\_progress=True}} in \sphinxcode{\sphinxupquote{dfols.solve()}}. If we do this for the above example, we get
\begin{quote}

\begin{sphinxVerbatim}[commandchars=\\\{\}]
 Run  Iter     Obj       Grad     Delta      rho     Evals
  1     1    1.43e+01  1.61e+02  1.20e\PYGZhy{}01  1.20e\PYGZhy{}01    3
  1     2    4.35e+00  3.77e+01  4.80e\PYGZhy{}01  1.20e\PYGZhy{}01    4
  1     3    4.35e+00  3.77e+01  6.00e\PYGZhy{}02  1.20e\PYGZhy{}02    4
...
  1    55    1.00e\PYGZhy{}02  2.00e\PYGZhy{}01  1.50e\PYGZhy{}08  1.00e\PYGZhy{}08   56
  1    56    1.00e\PYGZhy{}02  2.00e\PYGZhy{}01  1.50e\PYGZhy{}08  1.00e\PYGZhy{}08   57
\end{sphinxVerbatim}
\end{quote}


\section{Adding General Convex Constraints}
\label{\detokenize{userguide:adding-general-convex-constraints}}
\sphinxAtStartPar
We can also add more general convex constraints \(x \in C := C_1 \cap \cdots \cap C_n\) to our problem, where
each \(C_i\) is a convex set. To do this, we need to know the Euclidean projection operator for each \(C_i\):
\begin{equation*}
\begin{split}\operatorname{proj}_{C_i}(x) := \operatorname{argmin}_{y\in C_i} \|y-x\|_2^2.\end{split}
\end{equation*}
\sphinxAtStartPar
i.e. given a point \(x\), return the closest point to \(x\) in the set \(C_i\).
There are many examples of simple convex sets \(C_i\) for which this function has a known, simple form, such as:
\begin{itemize}
\item {} 
\sphinxAtStartPar
Bound constraints (but since DFO\sphinxhyphen{}LS supports this directly, it is better to give these explicitly via the \sphinxcode{\sphinxupquote{bounds}} input, as above)

\item {} 
\sphinxAtStartPar
Euclidean ball constraints: \(\|x-c\|_2 \leq r\)

\item {} 
\sphinxAtStartPar
Unit simplex: \(x_i \geq 0\) and \(\sum_{i=1}^{n} x_i \leq 1\)

\item {} 
\sphinxAtStartPar
Linear inequalities: \(a^T x \geq b\)

\end{itemize}

\sphinxAtStartPar
Note the intersection of the user\sphinxhyphen{}provided convex sets must be non\sphinxhyphen{}empty.

\sphinxAtStartPar
In DFO\sphinxhyphen{}LS, set the input \sphinxcode{\sphinxupquote{projections}} to be a list of projection functions, one per \(C_i\).
Internally, DFO\sphinxhyphen{}LS computes the projection onto the intersection of these sets and the bound constraints
using \sphinxhref{https://en.wikipedia.org/wiki/Dykstra\%27s\_projection\_algorithm}{Dykstra’s projection algorithm}.

\sphinxAtStartPar
For the explicit expressions for the above projections, and more examples, see for example \sphinxhref{https://proximity-operator.net/indicatorfunctions.html}{this online database}
or Section 6.4.6 of the textbook \sphinxcite{userguide:b2017}.

\sphinxAtStartPar
As an example, let’s minimize the above Rosenbrock function with different bounds, and with a Euclidean
ball constraint, namely \((x_1-0.7)^2 + (x_2-1.5)^2 \leq 0.4^2\).
\begin{quote}

\begin{sphinxVerbatim}[commandchars=\\\{\}]
\PYG{k+kn}{import} \PYG{n+nn}{numpy} \PYG{k}{as} \PYG{n+nn}{np}
\PYG{k+kn}{import} \PYG{n+nn}{dfols}

\PYG{c+c1}{\PYGZsh{} Define the objective function}
\PYG{k}{def} \PYG{n+nf}{rosenbrock}\PYG{p}{(}\PYG{n}{x}\PYG{p}{)}\PYG{p}{:}
    \PYG{k}{return} \PYG{n}{np}\PYG{o}{.}\PYG{n}{array}\PYG{p}{(}\PYG{p}{[}\PYG{l+m+mf}{10.0} \PYG{o}{*} \PYG{p}{(}\PYG{n}{x}\PYG{p}{[}\PYG{l+m+mi}{1}\PYG{p}{]} \PYG{o}{\PYGZhy{}} \PYG{n}{x}\PYG{p}{[}\PYG{l+m+mi}{0}\PYG{p}{]} \PYG{o}{*}\PYG{o}{*} \PYG{l+m+mi}{2}\PYG{p}{)}\PYG{p}{,} \PYG{l+m+mf}{1.0} \PYG{o}{\PYGZhy{}} \PYG{n}{x}\PYG{p}{[}\PYG{l+m+mi}{0}\PYG{p}{]}\PYG{p}{]}\PYG{p}{)}

\PYG{c+c1}{\PYGZsh{} Define the starting point}
\PYG{n}{x0} \PYG{o}{=} \PYG{n}{np}\PYG{o}{.}\PYG{n}{array}\PYG{p}{(}\PYG{p}{[}\PYG{o}{\PYGZhy{}}\PYG{l+m+mf}{1.2}\PYG{p}{,} \PYG{l+m+mi}{1}\PYG{p}{]}\PYG{p}{)}

\PYG{c+c1}{\PYGZsh{} Define bound constraints (lower \PYGZlt{}= x \PYGZlt{}= upper)}
\PYG{n}{lower} \PYG{o}{=} \PYG{n}{np}\PYG{o}{.}\PYG{n}{array}\PYG{p}{(}\PYG{p}{[}\PYG{o}{\PYGZhy{}}\PYG{l+m+mf}{2.0}\PYG{p}{,} \PYG{l+m+mf}{1.1}\PYG{p}{]}\PYG{p}{)}
\PYG{n}{upper} \PYG{o}{=} \PYG{n}{np}\PYG{o}{.}\PYG{n}{array}\PYG{p}{(}\PYG{p}{[}\PYG{l+m+mf}{0.9}\PYG{p}{,} \PYG{l+m+mf}{3.0}\PYG{p}{]}\PYG{p}{)}

\PYG{c+c1}{\PYGZsh{} Define the ball constraint ||x\PYGZhy{}center|| \PYGZlt{}= radius, and its projection operator}
\PYG{n}{center} \PYG{o}{=} \PYG{n}{np}\PYG{o}{.}\PYG{n}{array}\PYG{p}{(}\PYG{p}{[}\PYG{l+m+mf}{0.7}\PYG{p}{,} \PYG{l+m+mf}{1.5}\PYG{p}{]}\PYG{p}{)}
\PYG{n}{radius} \PYG{o}{=} \PYG{l+m+mf}{0.4}
\PYG{n}{ball\PYGZus{}proj} \PYG{o}{=} \PYG{k}{lambda} \PYG{n}{x}\PYG{p}{:} \PYG{n}{center} \PYG{o}{+} \PYG{p}{(}\PYG{n}{radius}\PYG{o}{/}\PYG{n+nb}{max}\PYG{p}{(}\PYG{n}{np}\PYG{o}{.}\PYG{n}{linalg}\PYG{o}{.}\PYG{n}{norm}\PYG{p}{(}\PYG{n}{x}\PYG{o}{\PYGZhy{}}\PYG{n}{center}\PYG{p}{)}\PYG{p}{,} \PYG{n}{radius}\PYG{p}{)}\PYG{p}{)} \PYG{o}{*} \PYG{p}{(}\PYG{n}{x}\PYG{o}{\PYGZhy{}}\PYG{n}{center}\PYG{p}{)}

\PYG{c+c1}{\PYGZsh{} Call DFO\PYGZhy{}LS (with bounds and projection operator)}
\PYG{c+c1}{\PYGZsh{} Note: it is better to provide bounds explicitly, instead of using the corresponding}
\PYG{c+c1}{\PYGZsh{}       projection function}
\PYG{c+c1}{\PYGZsh{} Note: input \PYGZsq{}projections\PYGZsq{} must be a list of projection functions}
\PYG{n}{soln} \PYG{o}{=} \PYG{n}{dfols}\PYG{o}{.}\PYG{n}{solve}\PYG{p}{(}\PYG{n}{rosenbrock}\PYG{p}{,} \PYG{n}{x0}\PYG{p}{,} \PYG{n}{bounds}\PYG{o}{=}\PYG{p}{(}\PYG{n}{lower}\PYG{p}{,} \PYG{n}{upper}\PYG{p}{)}\PYG{p}{,} \PYG{n}{projections}\PYG{o}{=}\PYG{p}{[}\PYG{n}{ball\PYGZus{}proj}\PYG{p}{]}\PYG{p}{)}

\PYG{c+c1}{\PYGZsh{} Display output}
\PYG{n+nb}{print}\PYG{p}{(}\PYG{n}{soln}\PYG{p}{)}
\end{sphinxVerbatim}
\end{quote}

\sphinxAtStartPar
Note that for bound constraints one can choose to either implement them by defining a projection function as above, or by passing the bounds as input like in the example from the section on adding bound constraints.

\sphinxAtStartPar
DFO\sphinxhyphen{}LS correctly finds the solution to this constrained problem too. Note that we get a warning because the step computed in the trust region subproblem
gave an increase in the model. This is common in the case where multiple constraints are active at the optimal point.
\begin{quote}

\begin{sphinxVerbatim}[commandchars=\\\{\}]
****** DFO\PYGZhy{}LS Results ******
Solution xmin = [0.9        1.15359245]
Residual vector = [3.43592448 0.1       ]
Objective value f(xmin) = 11.81557703
Needed 10 objective evaluations (at 10 points)
Approximate Jacobian = [[\PYGZhy{}1.79826221e+01  1.00004412e+01]
 [\PYGZhy{}1.00000000e+00 \PYGZhy{}1.81976605e\PYGZhy{}15]]
Solution xmin was evaluation point 5
Approximate Jacobian formed using evaluation points [8 5 9]
Exit flag = 5
Warning (trust region increase): Either multiple constraints are active or trust region step gave model increase
****************************
\end{sphinxVerbatim}
\end{quote}

\sphinxAtStartPar
Just like for bound constraints, DFO\sphinxhyphen{}LS will automatically ensure the starting point is feasible with respect to all constraints (bounds and general convex constraints).


\section{Adding a Regularizer}
\label{\detokenize{userguide:adding-a-regularizer}}
\sphinxAtStartPar
We can add a convex, Lipschitz continuous, but potentially non\sphinxhyphen{}differentiable regularizer to our objective function, to encourage the solution \(x\) to have certain properties.
This is most commonly used to avoid overfitting.
A very common choice of regularizer is \(h(x)=\lambda\|x\|_2^2\) for \(\lambda>0\) (called Tikhonov regularization or ridge regression), but this is not Lipschitz continuous. For this regularizer, you can add a new residual function \(r_{m+1}(x)=\sqrt{\lambda}\|x\|_2\) to the objective.

\sphinxAtStartPar
A suitable and widely used regularizer is the L1 norm (i.e. L1 regularization or LASSO), \(h(x)=\lambda\|x\|_1\) for \(\lambda>0\).
This encourages the solution \(x\) to be sparse (i.e. many entries are zero).
To use \(h(x)\) in DFO\sphinxhyphen{}LS, we need to know its \sphinxhref{https://en.wikipedia.org/wiki/Lipschitz\_continuity}{Lipschitz constant} and \sphinxhref{https://en.wikipedia.org/wiki/Proximal\_operator}{proximal operator}.

\sphinxAtStartPar
In this case, the Lipschitz constant of \(h(x)\) may be computed via
\begin{equation*}
\begin{split}|h(x) - h(x)| = \lambda\|x\|_1 - \lambda\|y\|_1 \leq \lambda\|x-y\|_1 \leq \lambda\sqrt{n} \|x-y\|_2\end{split}
\end{equation*}
\sphinxAtStartPar
using the reverse triangle inequality to get the first inequality. Hence the Lipschitz constant of \(h(x)\) is \(\lambda\sqrt{n}\).

\sphinxAtStartPar
The proximal operator for \(h(x)\) with a parameter \(u>0\) is defined as
\begin{equation*}
\begin{split}\operatorname{prox}_{uh}(x) := \operatorname{argmin}_{y\in\mathbb{R}^n} h(y) + \frac{1}{2u}\|y-x\|_2^2\end{split}
\end{equation*}
\sphinxAtStartPar
There are many regularizers with known proximal operators. See for example \sphinxhref{https://proximity-operator.net/}{this online database}
or Section 6.9 of the textbook \sphinxcite{userguide:b2017}.
In the case of \(h(x)=\lambda\|x\|_1\), the proximal operator is the soft\sphinxhyphen{}thresholding function, defined element\sphinxhyphen{}wise as
\begin{equation*}
\begin{split}[\operatorname{prox}_{uh}(x)]_i = \max(|x_i|-\lambda u, 0) \operatorname{sign}(x_i)\end{split}
\end{equation*}
\sphinxAtStartPar
We can use DFO\sphinxhyphen{}LS to solve a simple regularized linear least\sphinxhyphen{}squares problem (with artificially generated data) as follows:
\begin{quote}

\begin{sphinxVerbatim}[commandchars=\\\{\}]
\PYG{c+c1}{\PYGZsh{} DFO\PYGZhy{}LS example: regularized least\PYGZhy{}squares regression}
\PYG{k+kn}{import} \PYG{n+nn}{numpy} \PYG{k}{as} \PYG{n+nn}{np}
\PYG{k+kn}{import} \PYG{n+nn}{dfols}

\PYG{n}{n} \PYG{o}{=} \PYG{l+m+mi}{5}  \PYG{c+c1}{\PYGZsh{} dimension of x}
\PYG{n}{m} \PYG{o}{=} \PYG{l+m+mi}{10}  \PYG{c+c1}{\PYGZsh{} number of residuals, i.e. dimension of b}

\PYG{c+c1}{\PYGZsh{} Generate some artificial data for A and b}
\PYG{n}{A} \PYG{o}{=} \PYG{n}{np}\PYG{o}{.}\PYG{n}{arange}\PYG{p}{(}\PYG{n}{m}\PYG{o}{*}\PYG{n}{n}\PYG{p}{)}\PYG{o}{.}\PYG{n}{reshape}\PYG{p}{(}\PYG{p}{(}\PYG{n}{m}\PYG{p}{,}\PYG{n}{n}\PYG{p}{)}\PYG{p}{)}
\PYG{n}{b} \PYG{o}{=} \PYG{n}{np}\PYG{o}{.}\PYG{n}{sqrt}\PYG{p}{(}\PYG{n}{np}\PYG{o}{.}\PYG{n}{arange}\PYG{p}{(}\PYG{n}{m}\PYG{p}{)}\PYG{p}{)}
\PYG{n}{objfun} \PYG{o}{=} \PYG{k}{lambda} \PYG{n}{x}\PYG{p}{:} \PYG{n}{A} \PYG{o}{@} \PYG{n}{x} \PYG{o}{\PYGZhy{}} \PYG{n}{b}

\PYG{c+c1}{\PYGZsh{} L1 regularizer: h(x) = lda*||x||\PYGZus{}1 for some lda\PYGZgt{}0}
\PYG{n}{lda} \PYG{o}{=} \PYG{l+m+mf}{1.0}
\PYG{n}{h} \PYG{o}{=} \PYG{k}{lambda} \PYG{n}{x}\PYG{p}{:} \PYG{n}{lda} \PYG{o}{*} \PYG{n}{np}\PYG{o}{.}\PYG{n}{linalg}\PYG{o}{.}\PYG{n}{norm}\PYG{p}{(}\PYG{n}{x}\PYG{p}{,} \PYG{l+m+mi}{1}\PYG{p}{)}
\PYG{n}{Lh} \PYG{o}{=} \PYG{n}{lda} \PYG{o}{*} \PYG{n}{np}\PYG{o}{.}\PYG{n}{sqrt}\PYG{p}{(}\PYG{n}{n}\PYG{p}{)}  \PYG{c+c1}{\PYGZsh{} Lipschitz constant of h(x)}
\PYG{n}{prox\PYGZus{}uh} \PYG{o}{=} \PYG{k}{lambda} \PYG{n}{x}\PYG{p}{,} \PYG{n}{u}\PYG{p}{:} \PYG{n}{np}\PYG{o}{.}\PYG{n}{sign}\PYG{p}{(}\PYG{n}{x}\PYG{p}{)} \PYG{o}{*} \PYG{n}{np}\PYG{o}{.}\PYG{n}{maximum}\PYG{p}{(}\PYG{n}{np}\PYG{o}{.}\PYG{n}{abs}\PYG{p}{(}\PYG{n}{x}\PYG{p}{)} \PYG{o}{\PYGZhy{}} \PYG{n}{lda}\PYG{o}{*}\PYG{n}{u}\PYG{p}{,} \PYG{l+m+mf}{0.0}\PYG{p}{)}


\PYG{n}{x0} \PYG{o}{=} \PYG{n}{np}\PYG{o}{.}\PYG{n}{zeros}\PYG{p}{(}\PYG{p}{(}\PYG{n}{n}\PYG{p}{,}\PYG{p}{)}\PYG{p}{)}  \PYG{c+c1}{\PYGZsh{} arbitrary starting point}

\PYG{c+c1}{\PYGZsh{} Call DFO\PYGZhy{}LS}
\PYG{n}{soln} \PYG{o}{=} \PYG{n}{dfols}\PYG{o}{.}\PYG{n}{solve}\PYG{p}{(}\PYG{n}{objfun}\PYG{p}{,} \PYG{n}{x0}\PYG{p}{,} \PYG{n}{h}\PYG{o}{=}\PYG{n}{h}\PYG{p}{,} \PYG{n}{lh}\PYG{o}{=}\PYG{n}{Lh}\PYG{p}{,} \PYG{n}{prox\PYGZus{}uh}\PYG{o}{=}\PYG{n}{prox\PYGZus{}uh}\PYG{p}{)}

\PYG{c+c1}{\PYGZsh{} Display output}
\PYG{n+nb}{print}\PYG{p}{(}\PYG{n}{soln}\PYG{p}{)}
\end{sphinxVerbatim}
\end{quote}

\sphinxAtStartPar
The solution found by DFO\sphinxhyphen{}LS is:
\begin{quote}

\begin{sphinxVerbatim}[commandchars=\\\{\}]
****** DFO\PYGZhy{}LS Results ******
Solution xmin = [\PYGZhy{}6.85049254e\PYGZhy{}02 \PYGZhy{}7.03534168e\PYGZhy{}11  1.19957812e\PYGZhy{}15  7.47953030e\PYGZhy{}11
  1.30074165e\PYGZhy{}01]
Residual vector = [ 0.52029666 \PYGZhy{}0.17185715 \PYGZhy{}0.27822451 \PYGZhy{}0.28821556 \PYGZhy{}0.24831856 \PYGZhy{}0.17654034
 \PYGZhy{}0.08211591  0.02946872  0.1546391   0.29091242]
Objective value f(xmin) = 0.8682829845
Needed 34 objective evaluations (at 34 points)
Approximate Jacobian = [[\PYGZhy{}1.75619848e\PYGZhy{}09  1.00000000e+00  2.00000000e+00  3.00000000e+00
   4.00000000e+00]
 ...
 [ 4.50000000e+01  4.60000000e+01  4.70000000e+01  4.80000000e+01
   4.90000000e+01]]
Solution xmin was evaluation point 34
Approximate Jacobian formed using evaluation points [30 32 29 31 33 27]
Exit flag = 0
Success: rho has reached rhoend
****************************
\end{sphinxVerbatim}
\end{quote}

\sphinxAtStartPar
We can see that 3 of the 5 components of the solution are very close to zero.
Note that many LASSO\sphinxhyphen{}type algorithms can produce a solution with many entries being exactly zero, but DFO\sphinxhyphen{}LS can only make them very small (related to how it calculates a new point with trust\sphinxhyphen{}region constraints).


\section{Example: Noisy Objective Evaluation}
\label{\detokenize{userguide:example-noisy-objective-evaluation}}
\sphinxAtStartPar
As described in {\hyperref[\detokenize{info::doc}]{\sphinxcrossref{\DUrole{doc}{Overview}}}}, derivative\sphinxhyphen{}free algorithms such as DFO\sphinxhyphen{}LS are particularly useful when \sphinxcode{\sphinxupquote{objfun}} has noise. Let’s modify the previous example to include random noise in our objective evaluation, and compare it to SciPy’s derivative\sphinxhyphen{}based solver (the below results came from using SciPy v1.13.0):
\begin{quote}

\begin{sphinxVerbatim}[commandchars=\\\{\}]
\PYG{c+c1}{\PYGZsh{} DFO\PYGZhy{}LS example: minimize the noisy Rosenbrock function}
\PYG{k+kn}{from} \PYG{n+nn}{\PYGZus{}\PYGZus{}future\PYGZus{}\PYGZus{}} \PYG{k+kn}{import} \PYG{n}{print\PYGZus{}function}
\PYG{k+kn}{import} \PYG{n+nn}{numpy} \PYG{k}{as} \PYG{n+nn}{np}
\PYG{k+kn}{import} \PYG{n+nn}{dfols}

\PYG{c+c1}{\PYGZsh{} Define the objective function}
\PYG{k}{def} \PYG{n+nf}{rosenbrock}\PYG{p}{(}\PYG{n}{x}\PYG{p}{)}\PYG{p}{:}
    \PYG{k}{return} \PYG{n}{np}\PYG{o}{.}\PYG{n}{array}\PYG{p}{(}\PYG{p}{[}\PYG{l+m+mf}{10.0} \PYG{o}{*} \PYG{p}{(}\PYG{n}{x}\PYG{p}{[}\PYG{l+m+mi}{1}\PYG{p}{]} \PYG{o}{\PYGZhy{}} \PYG{n}{x}\PYG{p}{[}\PYG{l+m+mi}{0}\PYG{p}{]} \PYG{o}{*}\PYG{o}{*} \PYG{l+m+mi}{2}\PYG{p}{)}\PYG{p}{,} \PYG{l+m+mf}{1.0} \PYG{o}{\PYGZhy{}} \PYG{n}{x}\PYG{p}{[}\PYG{l+m+mi}{0}\PYG{p}{]}\PYG{p}{]}\PYG{p}{)}

\PYG{c+c1}{\PYGZsh{} Modified objective function: add 1\PYGZpc{} Gaussian noise}
\PYG{k}{def} \PYG{n+nf}{rosenbrock\PYGZus{}noisy}\PYG{p}{(}\PYG{n}{x}\PYG{p}{)}\PYG{p}{:}
    \PYG{k}{return} \PYG{n}{rosenbrock}\PYG{p}{(}\PYG{n}{x}\PYG{p}{)} \PYG{o}{*} \PYG{p}{(}\PYG{l+m+mf}{1.0} \PYG{o}{+} \PYG{l+m+mf}{1e\PYGZhy{}2} \PYG{o}{*} \PYG{n}{np}\PYG{o}{.}\PYG{n}{random}\PYG{o}{.}\PYG{n}{normal}\PYG{p}{(}\PYG{n}{size}\PYG{o}{=}\PYG{p}{(}\PYG{l+m+mi}{2}\PYG{p}{,}\PYG{p}{)}\PYG{p}{)}\PYG{p}{)}

\PYG{c+c1}{\PYGZsh{} Define the starting point}
\PYG{n}{x0} \PYG{o}{=} \PYG{n}{np}\PYG{o}{.}\PYG{n}{array}\PYG{p}{(}\PYG{p}{[}\PYG{o}{\PYGZhy{}}\PYG{l+m+mf}{1.2}\PYG{p}{,} \PYG{l+m+mf}{1.0}\PYG{p}{]}\PYG{p}{)}

\PYG{c+c1}{\PYGZsh{} Set random seed (for reproducibility)}
\PYG{n}{np}\PYG{o}{.}\PYG{n}{random}\PYG{o}{.}\PYG{n}{seed}\PYG{p}{(}\PYG{l+m+mi}{0}\PYG{p}{)}

\PYG{n+nb}{print}\PYG{p}{(}\PYG{l+s+s2}{\PYGZdq{}}\PYG{l+s+s2}{Demonstrate noise in function evaluation:}\PYG{l+s+s2}{\PYGZdq{}}\PYG{p}{)}
\PYG{k}{for} \PYG{n}{i} \PYG{o+ow}{in} \PYG{n+nb}{range}\PYG{p}{(}\PYG{l+m+mi}{5}\PYG{p}{)}\PYG{p}{:}
    \PYG{n+nb}{print}\PYG{p}{(}\PYG{l+s+s2}{\PYGZdq{}}\PYG{l+s+s2}{objfun(x0) = }\PYG{l+s+si}{\PYGZpc{}s}\PYG{l+s+s2}{\PYGZdq{}} \PYG{o}{\PYGZpc{}} \PYG{n+nb}{str}\PYG{p}{(}\PYG{n}{rosenbrock\PYGZus{}noisy}\PYG{p}{(}\PYG{n}{x0}\PYG{p}{)}\PYG{p}{)}\PYG{p}{)}
\PYG{n+nb}{print}\PYG{p}{(}\PYG{l+s+s2}{\PYGZdq{}}\PYG{l+s+s2}{\PYGZdq{}}\PYG{p}{)}

\PYG{c+c1}{\PYGZsh{} Call DFO\PYGZhy{}LS}
\PYG{n}{soln} \PYG{o}{=} \PYG{n}{dfols}\PYG{o}{.}\PYG{n}{solve}\PYG{p}{(}\PYG{n}{rosenbrock\PYGZus{}noisy}\PYG{p}{,} \PYG{n}{x0}\PYG{p}{)}

\PYG{c+c1}{\PYGZsh{} Display output}
\PYG{n+nb}{print}\PYG{p}{(}\PYG{n}{soln}\PYG{p}{)}

\PYG{c+c1}{\PYGZsh{} Compare with a derivative\PYGZhy{}based solver}
\PYG{k+kn}{import} \PYG{n+nn}{scipy}\PYG{n+nn}{.}\PYG{n+nn}{optimize} \PYG{k}{as} \PYG{n+nn}{opt}
\PYG{n}{soln} \PYG{o}{=} \PYG{n}{opt}\PYG{o}{.}\PYG{n}{least\PYGZus{}squares}\PYG{p}{(}\PYG{n}{rosenbrock\PYGZus{}noisy}\PYG{p}{,} \PYG{n}{x0}\PYG{p}{)}

\PYG{n+nb}{print}\PYG{p}{(}\PYG{l+s+s2}{\PYGZdq{}}\PYG{l+s+s2}{\PYGZdq{}}\PYG{p}{)}
\PYG{n+nb}{print}\PYG{p}{(}\PYG{l+s+s2}{\PYGZdq{}}\PYG{l+s+s2}{** SciPy results **}\PYG{l+s+s2}{\PYGZdq{}}\PYG{p}{)}
\PYG{n+nb}{print}\PYG{p}{(}\PYG{l+s+s2}{\PYGZdq{}}\PYG{l+s+s2}{Solution xmin = }\PYG{l+s+si}{\PYGZpc{}s}\PYG{l+s+s2}{\PYGZdq{}} \PYG{o}{\PYGZpc{}} \PYG{n+nb}{str}\PYG{p}{(}\PYG{n}{soln}\PYG{o}{.}\PYG{n}{x}\PYG{p}{)}\PYG{p}{)}
\PYG{n+nb}{print}\PYG{p}{(}\PYG{l+s+s2}{\PYGZdq{}}\PYG{l+s+s2}{Objective value f(xmin) = }\PYG{l+s+si}{\PYGZpc{}.10g}\PYG{l+s+s2}{\PYGZdq{}} \PYG{o}{\PYGZpc{}} \PYG{p}{(}\PYG{l+m+mf}{2.0} \PYG{o}{*} \PYG{n}{soln}\PYG{o}{.}\PYG{n}{cost}\PYG{p}{)}\PYG{p}{)}
\PYG{n+nb}{print}\PYG{p}{(}\PYG{l+s+s2}{\PYGZdq{}}\PYG{l+s+s2}{Needed }\PYG{l+s+si}{\PYGZpc{}g}\PYG{l+s+s2}{ objective evaluations}\PYG{l+s+s2}{\PYGZdq{}} \PYG{o}{\PYGZpc{}} \PYG{n}{soln}\PYG{o}{.}\PYG{n}{nfev}\PYG{p}{)}
\PYG{n+nb}{print}\PYG{p}{(}\PYG{l+s+s2}{\PYGZdq{}}\PYG{l+s+s2}{Exit flag = }\PYG{l+s+si}{\PYGZpc{}g}\PYG{l+s+s2}{\PYGZdq{}} \PYG{o}{\PYGZpc{}} \PYG{n}{soln}\PYG{o}{.}\PYG{n}{status}\PYG{p}{)}
\PYG{n+nb}{print}\PYG{p}{(}\PYG{n}{soln}\PYG{o}{.}\PYG{n}{message}\PYG{p}{)}
\end{sphinxVerbatim}
\end{quote}

\sphinxAtStartPar
The output of this is:
\begin{quote}

\begin{sphinxVerbatim}[commandchars=\\\{\}]
Demonstrate noise in function evaluation:
objfun(x0) = [\PYGZhy{}4.4776183   2.20880346]
objfun(x0) = [\PYGZhy{}4.44306447  2.24929965]
objfun(x0) = [\PYGZhy{}4.48217255  2.17849989]
objfun(x0) = [\PYGZhy{}4.44180389  2.19667014]
objfun(x0) = [\PYGZhy{}4.39545837  2.20903317]

****** DFO\PYGZhy{}LS Results ******
Solution xmin = [1.00000001 1.00000002]
Residual vector = [ 5.17481720e\PYGZhy{}09 \PYGZhy{}1.04150014e\PYGZhy{}08]
Objective value f(xmin) = 1.352509879e\PYGZhy{}16
Needed 35 objective evaluations (at 35 points)
Approximate Jacobian = [[\PYGZhy{}1.98079840e+01  1.00105722e+01]
 [\PYGZhy{}9.93887907e\PYGZhy{}01 \PYGZhy{}3.06567570e\PYGZhy{}04]]
Solution xmin was evaluation point 35
Approximate Jacobian formed using evaluation points [30 33 34]
Exit flag = 0
Success: Objective is sufficiently small
****************************


** SciPy results **
Solution xmin = [\PYGZhy{}1.2  1. ]
Objective value f(xmin) = 23.83907501
Needed 5 objective evaluations
Exit flag = 3
`xtol` termination condition is satisfied.
\end{sphinxVerbatim}
\end{quote}

\sphinxAtStartPar
DFO\sphinxhyphen{}LS is able to find the solution, but SciPy’s derivative\sphinxhyphen{}based solver, which has no trouble solving the noise\sphinxhyphen{}free problem, is unable to make any progress.

\sphinxAtStartPar
As noted above, DFO\sphinxhyphen{}LS has an input parameter \sphinxcode{\sphinxupquote{objfun\_has\_noise}} to indicate if \sphinxcode{\sphinxupquote{objfun}} has noise in it, which it does in this case. Therefore we can call DFO\sphinxhyphen{}LS with
\begin{quote}

\begin{sphinxVerbatim}[commandchars=\\\{\}]
\PYG{n}{soln} \PYG{o}{=} \PYG{n}{dfols}\PYG{o}{.}\PYG{n}{solve}\PYG{p}{(}\PYG{n}{rosenbrock\PYGZus{}noisy}\PYG{p}{,} \PYG{n}{x0}\PYG{p}{,} \PYG{n}{objfun\PYGZus{}has\PYGZus{}noise}\PYG{o}{=}\PYG{k+kc}{True}\PYG{p}{)}
\end{sphinxVerbatim}
\end{quote}

\sphinxAtStartPar
Using this setting, we find the correct solution faster:
\begin{quote}

\begin{sphinxVerbatim}[commandchars=\\\{\}]
****** DFO\PYGZhy{}LS Results ******
Solution xmin = [1. 1.]
Residual vector = [\PYGZhy{}6.56093684e\PYGZhy{}10 \PYGZhy{}1.17835345e\PYGZhy{}10]
Objective value f(xmin) = 4.443440912e\PYGZhy{}19
Needed 28 objective evaluations (at 28 points)
Approximate Jacobian = [[\PYGZhy{}1.98649933e+01  9.93403044e+00]
 [\PYGZhy{}9.93112150e\PYGZhy{}01  5.78830812e\PYGZhy{}03]]
Solution xmin was evaluation point 28
Approximate Jacobian formed using evaluation points [27 25 26]
Exit flag = 0
Success: Objective is sufficiently small
****************************
\end{sphinxVerbatim}
\end{quote}


\section{Example: Parameter Estimation/Data Fitting}
\label{\detokenize{userguide:example-parameter-estimation-data-fitting}}
\sphinxAtStartPar
Next, we show a short example of using DFO\sphinxhyphen{}LS to solve a parameter estimation problem (taken from \sphinxhref{https://uk.mathworks.com/help/optim/ug/lsqcurvefit.html\#examples}{here}). Given some observations \((t_i,y_i)\), we wish to calibrate parameters \(x=(x_1,x_2)\) in the exponential decay model
\begin{equation*}
\begin{split}y(t) = x_1 \exp(x_2 t)\end{split}
\end{equation*}
\sphinxAtStartPar
The code for this is:
\begin{quote}

\begin{sphinxVerbatim}[commandchars=\\\{\}]
\PYG{c+c1}{\PYGZsh{} DFO\PYGZhy{}LS example: data fitting problem}
\PYG{c+c1}{\PYGZsh{} Originally from:}
\PYG{c+c1}{\PYGZsh{} https://uk.mathworks.com/help/optim/ug/lsqcurvefit.html}
\PYG{k+kn}{from} \PYG{n+nn}{\PYGZus{}\PYGZus{}future\PYGZus{}\PYGZus{}} \PYG{k+kn}{import} \PYG{n}{print\PYGZus{}function}
\PYG{k+kn}{import} \PYG{n+nn}{numpy} \PYG{k}{as} \PYG{n+nn}{np}
\PYG{k+kn}{import} \PYG{n+nn}{dfols}

\PYG{c+c1}{\PYGZsh{} Observations}
\PYG{n}{tdata} \PYG{o}{=} \PYG{n}{np}\PYG{o}{.}\PYG{n}{array}\PYG{p}{(}\PYG{p}{[}\PYG{l+m+mf}{0.9}\PYG{p}{,} \PYG{l+m+mf}{1.5}\PYG{p}{,} \PYG{l+m+mf}{13.8}\PYG{p}{,} \PYG{l+m+mf}{19.8}\PYG{p}{,} \PYG{l+m+mf}{24.1}\PYG{p}{,} \PYG{l+m+mf}{28.2}\PYG{p}{,} \PYG{l+m+mf}{35.2}\PYG{p}{,}
                  \PYG{l+m+mf}{60.3}\PYG{p}{,} \PYG{l+m+mf}{74.6}\PYG{p}{,} \PYG{l+m+mf}{81.3}\PYG{p}{]}\PYG{p}{)}
\PYG{n}{ydata} \PYG{o}{=} \PYG{n}{np}\PYG{o}{.}\PYG{n}{array}\PYG{p}{(}\PYG{p}{[}\PYG{l+m+mf}{455.2}\PYG{p}{,} \PYG{l+m+mf}{428.6}\PYG{p}{,} \PYG{l+m+mf}{124.1}\PYG{p}{,} \PYG{l+m+mf}{67.3}\PYG{p}{,} \PYG{l+m+mf}{43.2}\PYG{p}{,} \PYG{l+m+mf}{28.1}\PYG{p}{,} \PYG{l+m+mf}{13.1}\PYG{p}{,}
                  \PYG{o}{\PYGZhy{}}\PYG{l+m+mf}{0.4}\PYG{p}{,} \PYG{o}{\PYGZhy{}}\PYG{l+m+mf}{1.3}\PYG{p}{,} \PYG{o}{\PYGZhy{}}\PYG{l+m+mf}{1.5}\PYG{p}{]}\PYG{p}{)}

\PYG{c+c1}{\PYGZsh{} Model is y(t) = x[0] * exp(x[1] * t)}
\PYG{k}{def} \PYG{n+nf}{prediction\PYGZus{}error}\PYG{p}{(}\PYG{n}{x}\PYG{p}{)}\PYG{p}{:}
    \PYG{k}{return} \PYG{n}{ydata} \PYG{o}{\PYGZhy{}} \PYG{n}{x}\PYG{p}{[}\PYG{l+m+mi}{0}\PYG{p}{]} \PYG{o}{*} \PYG{n}{np}\PYG{o}{.}\PYG{n}{exp}\PYG{p}{(}\PYG{n}{x}\PYG{p}{[}\PYG{l+m+mi}{1}\PYG{p}{]} \PYG{o}{*} \PYG{n}{tdata}\PYG{p}{)}

\PYG{c+c1}{\PYGZsh{} Define the starting point}
\PYG{n}{x0} \PYG{o}{=} \PYG{n}{np}\PYG{o}{.}\PYG{n}{array}\PYG{p}{(}\PYG{p}{[}\PYG{l+m+mf}{100.0}\PYG{p}{,} \PYG{o}{\PYGZhy{}}\PYG{l+m+mf}{1.0}\PYG{p}{]}\PYG{p}{)}

\PYG{c+c1}{\PYGZsh{} We expect exponential decay: set upper bound x[1] \PYGZlt{}= 0}
\PYG{n}{upper} \PYG{o}{=} \PYG{n}{np}\PYG{o}{.}\PYG{n}{array}\PYG{p}{(}\PYG{p}{[}\PYG{l+m+mf}{1e20}\PYG{p}{,} \PYG{l+m+mf}{0.0}\PYG{p}{]}\PYG{p}{)}

\PYG{c+c1}{\PYGZsh{} Call DFO\PYGZhy{}LS}
\PYG{n}{soln} \PYG{o}{=} \PYG{n}{dfols}\PYG{o}{.}\PYG{n}{solve}\PYG{p}{(}\PYG{n}{prediction\PYGZus{}error}\PYG{p}{,} \PYG{n}{x0}\PYG{p}{,} \PYG{n}{bounds}\PYG{o}{=}\PYG{p}{(}\PYG{k+kc}{None}\PYG{p}{,} \PYG{n}{upper}\PYG{p}{)}\PYG{p}{)}

\PYG{c+c1}{\PYGZsh{} Display output}
\PYG{n+nb}{print}\PYG{p}{(}\PYG{n}{soln}\PYG{p}{)}
\end{sphinxVerbatim}
\end{quote}

\sphinxAtStartPar
The output of this is (noting that DFO\sphinxhyphen{}LS moves \(x_0\) to be far away enough from the upper bound)
\begin{quote}

\begin{sphinxVerbatim}[commandchars=\\\{\}]
****** DFO\PYGZhy{}LS Results ******
Solution xmin = [ 4.98830861e+02 \PYGZhy{}1.01256863e\PYGZhy{}01]
Residual vector = [\PYGZhy{}0.1816709   0.06098396  0.76276296  0.11962351 \PYGZhy{}0.26589799 \PYGZhy{}0.59788816
 \PYGZhy{}1.02611898 \PYGZhy{}1.51235371 \PYGZhy{}1.56145452 \PYGZhy{}1.63266662]
Objective value f(xmin) = 9.504886892
Needed 111 objective evaluations (at 111 points)
Approximate Jacobian = [[\PYGZhy{}9.12901055e\PYGZhy{}01 \PYGZhy{}4.09843504e+02]
 [\PYGZhy{}8.59087363e\PYGZhy{}01 \PYGZhy{}6.42808534e+02]
 [\PYGZhy{}2.47254068e\PYGZhy{}01 \PYGZhy{}1.70205403e+03]
 [\PYGZhy{}1.34676757e\PYGZhy{}01 \PYGZhy{}1.33017163e+03]
 [\PYGZhy{}8.71358948e\PYGZhy{}02 \PYGZhy{}1.04752831e+03]
 [\PYGZhy{}5.75309286e\PYGZhy{}02 \PYGZhy{}8.09280596e+02]
 [\PYGZhy{}2.83185935e\PYGZhy{}02 \PYGZhy{}4.97239504e+02]
 [\PYGZhy{}2.22997879e\PYGZhy{}03 \PYGZhy{}6.70749550e+01]
 [\PYGZhy{}5.24146460e\PYGZhy{}04 \PYGZhy{}1.95045170e+01]
 [\PYGZhy{}2.65964661e\PYGZhy{}04 \PYGZhy{}1.07858021e+01]]
Solution xmin was evaluation point 111
Approximate Jacobian formed using evaluation points [104 109 110]
Exit flag = 0
Success: rho has reached rhoend
****************************
\end{sphinxVerbatim}
\end{quote}

\sphinxAtStartPar
This produces a good fit to the observations.

\noindent{\hspace*{\fill}\sphinxincludegraphics[width=0.750\linewidth]{{data_fitting}.png}\hspace*{\fill}}

\sphinxAtStartPar
To generate this plot, run:
\begin{quote}

\begin{sphinxVerbatim}[commandchars=\\\{\}]
\PYG{c+c1}{\PYGZsh{} Plot calibrated model vs. observations}
\PYG{n}{ts} \PYG{o}{=} \PYG{n}{np}\PYG{o}{.}\PYG{n}{linspace}\PYG{p}{(}\PYG{l+m+mf}{0.0}\PYG{p}{,} \PYG{l+m+mf}{90.0}\PYG{p}{)}
\PYG{n}{ys} \PYG{o}{=} \PYG{n}{soln}\PYG{o}{.}\PYG{n}{x}\PYG{p}{[}\PYG{l+m+mi}{0}\PYG{p}{]} \PYG{o}{*} \PYG{n}{np}\PYG{o}{.}\PYG{n}{exp}\PYG{p}{(}\PYG{n}{soln}\PYG{o}{.}\PYG{n}{x}\PYG{p}{[}\PYG{l+m+mi}{1}\PYG{p}{]} \PYG{o}{*} \PYG{n}{ts}\PYG{p}{)}

\PYG{k+kn}{import} \PYG{n+nn}{matplotlib}\PYG{n+nn}{.}\PYG{n+nn}{pyplot} \PYG{k}{as} \PYG{n+nn}{plt}
\PYG{n}{plt}\PYG{o}{.}\PYG{n}{figure}\PYG{p}{(}\PYG{l+m+mi}{1}\PYG{p}{)}
\PYG{n}{ax} \PYG{o}{=} \PYG{n}{plt}\PYG{o}{.}\PYG{n}{gca}\PYG{p}{(}\PYG{p}{)}  \PYG{c+c1}{\PYGZsh{} current axes}
\PYG{n}{ax}\PYG{o}{.}\PYG{n}{plot}\PYG{p}{(}\PYG{n}{ts}\PYG{p}{,} \PYG{n}{ys}\PYG{p}{,} \PYG{l+s+s1}{\PYGZsq{}}\PYG{l+s+s1}{k\PYGZhy{}}\PYG{l+s+s1}{\PYGZsq{}}\PYG{p}{,} \PYG{n}{label}\PYG{o}{=}\PYG{l+s+s1}{\PYGZsq{}}\PYG{l+s+s1}{Model}\PYG{l+s+s1}{\PYGZsq{}}\PYG{p}{)}
\PYG{n}{ax}\PYG{o}{.}\PYG{n}{plot}\PYG{p}{(}\PYG{n}{tdata}\PYG{p}{,} \PYG{n}{ydata}\PYG{p}{,} \PYG{l+s+s1}{\PYGZsq{}}\PYG{l+s+s1}{bo}\PYG{l+s+s1}{\PYGZsq{}}\PYG{p}{,} \PYG{n}{label}\PYG{o}{=}\PYG{l+s+s1}{\PYGZsq{}}\PYG{l+s+s1}{Data}\PYG{l+s+s1}{\PYGZsq{}}\PYG{p}{)}
\PYG{n}{ax}\PYG{o}{.}\PYG{n}{set\PYGZus{}xlabel}\PYG{p}{(}\PYG{l+s+s1}{\PYGZsq{}}\PYG{l+s+s1}{t}\PYG{l+s+s1}{\PYGZsq{}}\PYG{p}{)}
\PYG{n}{ax}\PYG{o}{.}\PYG{n}{set\PYGZus{}ylabel}\PYG{p}{(}\PYG{l+s+s1}{\PYGZsq{}}\PYG{l+s+s1}{y(t)}\PYG{l+s+s1}{\PYGZsq{}}\PYG{p}{)}
\PYG{n}{ax}\PYG{o}{.}\PYG{n}{legend}\PYG{p}{(}\PYG{n}{loc}\PYG{o}{=}\PYG{l+s+s1}{\PYGZsq{}}\PYG{l+s+s1}{upper right}\PYG{l+s+s1}{\PYGZsq{}}\PYG{p}{)}
\PYG{n}{ax}\PYG{o}{.}\PYG{n}{grid}\PYG{p}{(}\PYG{p}{)}
\PYG{n}{plt}\PYG{o}{.}\PYG{n}{show}\PYG{p}{(}\PYG{p}{)}
\end{sphinxVerbatim}
\end{quote}


\section{Example: Solving a Nonlinear System of Equations}
\label{\detokenize{userguide:example-solving-a-nonlinear-system-of-equations}}
\sphinxAtStartPar
Lastly, we give an example of using DFO\sphinxhyphen{}LS to solve a nonlinear system of equations (taken from \sphinxhref{http://support.sas.com/documentation/cdl/en/imlug/66112/HTML/default/viewer.htm\#imlug\_genstatexpls\_sect004.htm}{here}). We wish to solve the following set of equations
\begin{equation*}
\begin{split}x_1 + x_2 - x_1 x_2 + 2 &= 0, \\
x_1 \exp(-x_2) - 1 &= 0.\end{split}
\end{equation*}
\sphinxAtStartPar
The code for this is:
\begin{quote}

\begin{sphinxVerbatim}[commandchars=\\\{\}]
\PYG{c+c1}{\PYGZsh{} DFO\PYGZhy{}LS example: Solving a nonlinear system of equations}
\PYG{c+c1}{\PYGZsh{} Originally from:}
\PYG{c+c1}{\PYGZsh{} http://support.sas.com/documentation/cdl/en/imlug/66112/HTML/default/viewer.htm\PYGZsh{}imlug\PYGZus{}genstatexpls\PYGZus{}sect004.htm}

\PYG{k+kn}{from} \PYG{n+nn}{\PYGZus{}\PYGZus{}future\PYGZus{}\PYGZus{}} \PYG{k+kn}{import} \PYG{n}{print\PYGZus{}function}
\PYG{k+kn}{from} \PYG{n+nn}{math} \PYG{k+kn}{import} \PYG{n}{exp}
\PYG{k+kn}{import} \PYG{n+nn}{numpy} \PYG{k}{as} \PYG{n+nn}{np}
\PYG{k+kn}{import} \PYG{n+nn}{dfols}

\PYG{c+c1}{\PYGZsh{} Want to solve:}
\PYG{c+c1}{\PYGZsh{}   x1 + x2 \PYGZhy{} x1*x2 + 2 = 0}
\PYG{c+c1}{\PYGZsh{}   x1 * exp(\PYGZhy{}x2) \PYGZhy{} 1   = 0}
\PYG{k}{def} \PYG{n+nf}{nonlinear\PYGZus{}system}\PYG{p}{(}\PYG{n}{x}\PYG{p}{)}\PYG{p}{:}
    \PYG{k}{return} \PYG{n}{np}\PYG{o}{.}\PYG{n}{array}\PYG{p}{(}\PYG{p}{[}\PYG{n}{x}\PYG{p}{[}\PYG{l+m+mi}{0}\PYG{p}{]} \PYG{o}{+} \PYG{n}{x}\PYG{p}{[}\PYG{l+m+mi}{1}\PYG{p}{]} \PYG{o}{\PYGZhy{}} \PYG{n}{x}\PYG{p}{[}\PYG{l+m+mi}{0}\PYG{p}{]}\PYG{o}{*}\PYG{n}{x}\PYG{p}{[}\PYG{l+m+mi}{1}\PYG{p}{]} \PYG{o}{+} \PYG{l+m+mi}{2}\PYG{p}{,}
                     \PYG{n}{x}\PYG{p}{[}\PYG{l+m+mi}{0}\PYG{p}{]} \PYG{o}{*} \PYG{n}{exp}\PYG{p}{(}\PYG{o}{\PYGZhy{}}\PYG{n}{x}\PYG{p}{[}\PYG{l+m+mi}{1}\PYG{p}{]}\PYG{p}{)} \PYG{o}{\PYGZhy{}} \PYG{l+m+mf}{1.0}\PYG{p}{]}\PYG{p}{)}

\PYG{c+c1}{\PYGZsh{} Warning: if there are multiple solutions, which one}
\PYG{c+c1}{\PYGZsh{}          DFO\PYGZhy{}LS returns will likely depend on x0!}
\PYG{n}{x0} \PYG{o}{=} \PYG{n}{np}\PYG{o}{.}\PYG{n}{array}\PYG{p}{(}\PYG{p}{[}\PYG{l+m+mf}{0.1}\PYG{p}{,} \PYG{o}{\PYGZhy{}}\PYG{l+m+mf}{2.0}\PYG{p}{]}\PYG{p}{)}

\PYG{c+c1}{\PYGZsh{} Call DFO\PYGZhy{}LS}
\PYG{n}{soln} \PYG{o}{=} \PYG{n}{dfols}\PYG{o}{.}\PYG{n}{solve}\PYG{p}{(}\PYG{n}{nonlinear\PYGZus{}system}\PYG{p}{,} \PYG{n}{x0}\PYG{p}{)}

\PYG{c+c1}{\PYGZsh{} Display output}
\PYG{n+nb}{print}\PYG{p}{(}\PYG{n}{soln}\PYG{p}{)}
\end{sphinxVerbatim}
\end{quote}

\sphinxAtStartPar
The output of this is
\begin{quote}

\begin{sphinxVerbatim}[commandchars=\\\{\}]
****** DFO\PYGZhy{}LS Results ******
Solution xmin = [ 0.09777309 \PYGZhy{}2.32510588]
Residual vector = [\PYGZhy{}1.38601752e\PYGZhy{}09 \PYGZhy{}1.70204653e\PYGZhy{}08]
Objective value f(xmin) = 2.916172822e\PYGZhy{}16
Needed 13 objective evaluations (at 13 points)
Approximate Jacobian = [[ 3.32527052  0.90227531]
 [10.22943034 \PYGZhy{}0.99958226]]
Solution xmin was evaluation point 13
Approximate Jacobian formed using evaluation points [ 8 11 12]
Exit flag = 0
Success: Objective is sufficiently small
****************************
\end{sphinxVerbatim}
\end{quote}

\sphinxAtStartPar
Here, we see that both entries of the residual vector are very small, so both equations have been solved to high accuracy.


\section{References}
\label{\detokenize{userguide:references}}
\sphinxstepscope


\chapter{Advanced Usage}
\label{\detokenize{advanced:advanced-usage}}\label{\detokenize{advanced::doc}}
\sphinxAtStartPar
This section describes different optional user parameters available in DFO\sphinxhyphen{}LS.

\sphinxAtStartPar
In the last section ({\hyperref[\detokenize{userguide::doc}]{\sphinxcrossref{\DUrole{doc}{Using DFO\sphinxhyphen{}LS}}}}), we introduced \sphinxcode{\sphinxupquote{dfols.solve()}}, which has the optional input \sphinxcode{\sphinxupquote{user\_params}}. This is a Python dictionary of user parameters. We will now go through the settings which can be changed in this way. More details are available in the papers \sphinxcite{userguide:cfmr2018}, \sphinxcite{userguide:hr2022} and \sphinxcite{userguide:llr2024}.

\sphinxAtStartPar
The default values, used if no override is given, in some cases vary depending on whether \sphinxcode{\sphinxupquote{objfun}} has stochastic noise; that is, whether evaluating \sphinxcode{\sphinxupquote{objfun(x)}} several times at the same \sphinxcode{\sphinxupquote{x}} gives the same result or not. Whether or not this is the case is determined by the \sphinxcode{\sphinxupquote{objfun\_has\_noise}} input to \sphinxcode{\sphinxupquote{dfols.solve()}} (and not by inspecting \sphinxcode{\sphinxupquote{objfun}}, for instance).


\section{General Algorithm Parameters}
\label{\detokenize{advanced:general-algorithm-parameters}}\begin{itemize}
\item {} 
\sphinxAtStartPar
\sphinxcode{\sphinxupquote{general.rounding\_error\_constant}} \sphinxhyphen{} Internally, all interpolation points are stored with respect to a base point \(x_b\); that is, we store \(\{y_t-x_b\}\), which reduces the risk of roundoff errors. We shift \(x_b\) to \(x_k\) when \(\|s_k\| \leq \text{const}\|x_k-x_b\|\), where ‘const’ is this parameter. Default is 0.1.

\item {} 
\sphinxAtStartPar
\sphinxcode{\sphinxupquote{general.safety\_step\_thresh}} \sphinxhyphen{} Threshold for when to call the safety step, \(\|s_k\| \leq \gamma_S \rho_k\). Default is \(\gamma_S =0.5\).

\item {} 
\sphinxAtStartPar
\sphinxcode{\sphinxupquote{general.check\_objfun\_for\_overflow}} \sphinxhyphen{} Whether to cap the value of \(r_i(x)\) when they are large enough that an OverflowError will be encountered when trying to evaluate \(f(x)\). Default is \sphinxcode{\sphinxupquote{True}}.

\end{itemize}


\section{Logging and Output}
\label{\detokenize{advanced:logging-and-output}}\begin{itemize}
\item {} 
\sphinxAtStartPar
\sphinxcode{\sphinxupquote{logging.n\_to\_print\_whole\_x\_vector}} \sphinxhyphen{} If printing all function evaluations to screen/log file, the maximum \sphinxcode{\sphinxupquote{len(x)}} for which the full vector \sphinxcode{\sphinxupquote{x}} should be printed also. Default is 6.

\item {} 
\sphinxAtStartPar
\sphinxcode{\sphinxupquote{logging.save\_diagnostic\_info}} \sphinxhyphen{} Flag so save diagnostic information at each iteration. Default is \sphinxcode{\sphinxupquote{False}}.

\item {} 
\sphinxAtStartPar
\sphinxcode{\sphinxupquote{logging.save\_poisedness}} \sphinxhyphen{} If saving diagnostic information, whether to include the \(\Lambda\)\sphinxhyphen{}poisedness of \(Y_k\) in the diagnostic information. This is the most computationally expensive piece of diagnostic information. Default is \sphinxcode{\sphinxupquote{True}}.

\item {} 
\sphinxAtStartPar
\sphinxcode{\sphinxupquote{logging.save\_xk}} \sphinxhyphen{} If saving diagnostic information, whether to include the full vector \(x_k\). Default is \sphinxcode{\sphinxupquote{False}}.

\item {} 
\sphinxAtStartPar
\sphinxcode{\sphinxupquote{logging.save\_rk}} \sphinxhyphen{} If saving diagnostic information, whether to include the full vector \([r_1(x_k)\:\cdots\:r_m(x_k)]\). The value \(f(x_k)\) is always included. Default is \sphinxcode{\sphinxupquote{False}}.

\end{itemize}


\section{Initialization of Points}
\label{\detokenize{advanced:initialization-of-points}}\begin{itemize}
\item {} 
\sphinxAtStartPar
\sphinxcode{\sphinxupquote{init.random\_initial\_directions}} \sphinxhyphen{} Build the initial interpolation set using random directions (as opposed to coordinate directions). Default as of version 1.2 is \sphinxcode{\sphinxupquote{False}}.

\item {} 
\sphinxAtStartPar
\sphinxcode{\sphinxupquote{init.random\_directions\_make\_orthogonal}} \sphinxhyphen{} If building initial interpolation set with random directions, whether or not these should be orthogonalized. Default is \sphinxcode{\sphinxupquote{True}}.

\item {} 
\sphinxAtStartPar
\sphinxcode{\sphinxupquote{init.run\_in\_parallel}} \sphinxhyphen{} If using random directions or non\sphinxhyphen{}random with input \sphinxcode{\sphinxupquote{npt}} at most \sphinxcode{\sphinxupquote{len(x0)+1}}, whether or not to ask for all \sphinxcode{\sphinxupquote{objfun}} to be evaluated at all points without any intermediate processing. Default is \sphinxcode{\sphinxupquote{False}}.

\end{itemize}


\section{Trust Region Management}
\label{\detokenize{advanced:trust-region-management}}\begin{itemize}
\item {} 
\sphinxAtStartPar
\sphinxcode{\sphinxupquote{tr\_radius.eta1}} \sphinxhyphen{} Threshold for unsuccessful trust region iteration, \(\eta_1\). Default is 0.1.

\item {} 
\sphinxAtStartPar
\sphinxcode{\sphinxupquote{tr\_radius.eta2}} \sphinxhyphen{} Threshold for very successful trust region iteration, \(\eta_2\). Default is 0.7.

\item {} 
\sphinxAtStartPar
\sphinxcode{\sphinxupquote{tr\_radius.gamma\_dec}} \sphinxhyphen{} Ratio to decrease \(\Delta_k\) in unsuccessful iteration, \(\gamma_{dec}\). Default is 0.5 for smooth problems or 0.98 for noisy problems (i.e. \sphinxcode{\sphinxupquote{objfun\_has\_noise = True}}).

\item {} 
\sphinxAtStartPar
\sphinxcode{\sphinxupquote{tr\_radius.gamma\_inc}} \sphinxhyphen{} Ratio to increase \(\Delta_k\) in very successful iterations, \(\gamma_{inc}\). Default is 2.

\item {} 
\sphinxAtStartPar
\sphinxcode{\sphinxupquote{tr\_radius.gamma\_inc\_overline}} \sphinxhyphen{} Ratio of \(\|s_k\|\) to increase \(\Delta_k\) by in very successful iterations, \(\overline{\gamma}_{inc}\). Default is 4.

\item {} 
\sphinxAtStartPar
\sphinxcode{\sphinxupquote{tr\_radius.alpha1}} \sphinxhyphen{} Ratio to decrease \(\rho_k\) by when it is reduced, \(\alpha_1\). Default is 0.1 for smooth problems or 0.9 for noisy problems (i.e. \sphinxcode{\sphinxupquote{objfun\_has\_noise = True}}).

\item {} 
\sphinxAtStartPar
\sphinxcode{\sphinxupquote{tr\_radius.alpha2}} \sphinxhyphen{} Ratio of \(\rho_k\) to decrease \(\Delta_k\) by when \(\rho_k\) is reduced, \(\alpha_2\). Default is 0.5 for smooth problems or 0.95 for noisy problems (i.e. \sphinxcode{\sphinxupquote{objfun\_has\_noise = True}}).

\end{itemize}


\section{Termination on Small Objective Value}
\label{\detokenize{advanced:termination-on-small-objective-value}}\begin{itemize}
\item {} 
\sphinxAtStartPar
\sphinxcode{\sphinxupquote{model.abs\_tol}} \sphinxhyphen{} Tolerance on \(f(x_k)\); quit if \(f(x_k)\) is below this value. Default is \(10^{-12}\).

\item {} 
\sphinxAtStartPar
\sphinxcode{\sphinxupquote{model.rel\_tol}} \sphinxhyphen{} Relative tolerance on \(f(x_k)\); quit if \(f(x_k)/f(x_0)\) is below this value. Default is \(10^{-20}\).

\end{itemize}


\section{Termination on Slow Progress}
\label{\detokenize{advanced:termination-on-slow-progress}}\begin{itemize}
\item {} 
\sphinxAtStartPar
\sphinxcode{\sphinxupquote{slow.history\_for\_slow}} \sphinxhyphen{} History used to determine whether the current iteration is ‘slow’. Default is 5.

\item {} 
\sphinxAtStartPar
\sphinxcode{\sphinxupquote{slow.thresh\_for\_slow}} \sphinxhyphen{} Threshold for objective decrease used to determine whether the current iteration is ‘slow’. Default is \(10^{-4}\).

\item {} 
\sphinxAtStartPar
\sphinxcode{\sphinxupquote{slow.max\_slow\_iters}} \sphinxhyphen{} Number of consecutive slow successful iterations before termination (or restart). Default is \sphinxcode{\sphinxupquote{20*len(x0)}}.

\end{itemize}


\section{Stochastic Noise Information}
\label{\detokenize{advanced:stochastic-noise-information}}\begin{itemize}
\item {} 
\sphinxAtStartPar
\sphinxcode{\sphinxupquote{noise.quit\_on\_noise\_level}} \sphinxhyphen{} Flag to quit (or restart) if all \(f(y_t)\) are within noise level of \(f(x_k)\). Default is \sphinxcode{\sphinxupquote{False}} for smooth problems or \sphinxcode{\sphinxupquote{True}} for noisy problems.

\item {} 
\sphinxAtStartPar
\sphinxcode{\sphinxupquote{noise.scale\_factor\_for\_quit}} \sphinxhyphen{} Factor of noise level to use in termination criterion. Default is 1.

\item {} 
\sphinxAtStartPar
\sphinxcode{\sphinxupquote{noise.multiplicative\_noise\_level}} \sphinxhyphen{} Multiplicative noise level in \(f\). Can only specify one of multiplicative or additive noise levels. Default is \sphinxcode{\sphinxupquote{None}}.

\item {} 
\sphinxAtStartPar
\sphinxcode{\sphinxupquote{noise.additive\_noise\_level}} \sphinxhyphen{} Additive noise level in \(f\). Can only specify one of multiplicative or additive noise levels. Default is \sphinxcode{\sphinxupquote{None}}.

\end{itemize}


\section{Interpolation Management}
\label{\detokenize{advanced:interpolation-management}}\begin{itemize}
\item {} 
\sphinxAtStartPar
\sphinxcode{\sphinxupquote{interpolation.precondition}} \sphinxhyphen{} whether or not to scale the interpolation linear system to improve conditioning. Default is \sphinxcode{\sphinxupquote{True}}.

\item {} 
\sphinxAtStartPar
\sphinxcode{\sphinxupquote{interpolation.throw\_error\_on\_nans}} \sphinxhyphen{} whether or not to throw \sphinxcode{\sphinxupquote{numpy.linalg.LinAlgError}} if trying to interpolate to NaN objective values. If \sphinxcode{\sphinxupquote{False}}, DFO\sphinxhyphen{}LS should terminate gracefully with an error flag. Default is \sphinxcode{\sphinxupquote{False}}.

\end{itemize}


\section{Regression Model Management}
\label{\detokenize{advanced:regression-model-management}}\begin{itemize}
\item {} 
\sphinxAtStartPar
\sphinxcode{\sphinxupquote{regression.num\_extra\_steps}} \sphinxhyphen{} In successful iterations, the number of extra points (other than accepting the trust region step) to move, useful when \(|Y_k|>n+1\) (\(n\) is \sphinxcode{\sphinxupquote{len(x0)}}). Default is 0.

\item {} 
\sphinxAtStartPar
\sphinxcode{\sphinxupquote{regression.increase\_num\_extra\_steps\_with\_restart}} \sphinxhyphen{} The amount to increase \sphinxcode{\sphinxupquote{regression.num\_extra\_steps}} by with each restarts, for instance if increasing the number of points with each restart. Default is 0.

\item {} 
\sphinxAtStartPar
\sphinxcode{\sphinxupquote{regression.momentum\_extra\_steps}} \sphinxhyphen{} If moving extra points in successful iterations, whether to use the ‘momentum’ method. If not, uses geometry\sphinxhyphen{}improving steps. Default is \sphinxcode{\sphinxupquote{False}}.

\end{itemize}


\section{Multiple Restarts}
\label{\detokenize{advanced:multiple-restarts}}\begin{itemize}
\item {} 
\sphinxAtStartPar
\sphinxcode{\sphinxupquote{restarts.use\_restarts}} \sphinxhyphen{} Whether to do restarts when \(\rho_k\) reaches \(\rho_{end}\), or (optionally) when all points are within noise level of \(f(x_k)\). Default is \sphinxcode{\sphinxupquote{False}} for smooth problems or \sphinxcode{\sphinxupquote{True}} for noisy problems.

\item {} 
\sphinxAtStartPar
\sphinxcode{\sphinxupquote{restarts.max\_unsuccessful\_restarts}} \sphinxhyphen{} Maximum number of consecutive unsuccessful restarts allowed (i.e.\textasciitilde{}restarts which did not reduce the objective further). Default is 10.

\item {} 
\sphinxAtStartPar
\sphinxcode{\sphinxupquote{restarts.rhoend\_scale}} \sphinxhyphen{} Factor to reduce \(\rho_{end}\) by with each restart. Default is 1.

\item {} 
\sphinxAtStartPar
\sphinxcode{\sphinxupquote{restarts.use\_soft\_restarts}} \sphinxhyphen{} Whether to use soft or hard restarts. Default is \sphinxcode{\sphinxupquote{True}}.

\item {} 
\sphinxAtStartPar
\sphinxcode{\sphinxupquote{restarts.soft.num\_geom\_steps}} \sphinxhyphen{} For soft restarts, the number of points to move. Default is 3.

\item {} 
\sphinxAtStartPar
\sphinxcode{\sphinxupquote{restarts.soft.move\_xk}} \sphinxhyphen{} For soft restarts, whether to preserve \(x_k\), or move it to the best new point evaluated. Default is \sphinxcode{\sphinxupquote{True}}.

\item {} 
\sphinxAtStartPar
\sphinxcode{\sphinxupquote{restarts.increase\_npt}} \sphinxhyphen{} Whether to increase \(|Y_k|\) with each restart. Default is \sphinxcode{\sphinxupquote{False}}.

\item {} 
\sphinxAtStartPar
\sphinxcode{\sphinxupquote{restarts.increase\_npt\_amt}} \sphinxhyphen{} Amount to increase \(|Y_k|\) by with each restart. Default is 1.

\item {} 
\sphinxAtStartPar
\sphinxcode{\sphinxupquote{restarts.hard.increase\_ndirs\_initial\_amt}} \sphinxhyphen{} Amount to increase \sphinxcode{\sphinxupquote{growing.ndirs\_initial}} by with each hard restart. To avoid a growing phase, it is best to set it to the same value as \sphinxcode{\sphinxupquote{restarts.increase\_npt\_amt}}. Default is 1.

\item {} 
\sphinxAtStartPar
\sphinxcode{\sphinxupquote{restarts.hard.use\_old\_rk}} \sphinxhyphen{} If using hard restarts, whether or not to recycle the objective value at the best iterate found when performing a restart. This saves one objective evaluation. Default is \sphinxcode{\sphinxupquote{True}}.

\item {} 
\sphinxAtStartPar
\sphinxcode{\sphinxupquote{restarts.max\_npt}} \sphinxhyphen{} Maximum allowed value of \(|Y_k|\), useful if increasing with each restart. Default is \sphinxcode{\sphinxupquote{npt}}, the input parameter to \sphinxcode{\sphinxupquote{dfols.solve()}}.

\item {} 
\sphinxAtStartPar
\sphinxcode{\sphinxupquote{restarts.soft.max\_fake\_successful\_steps}} \sphinxhyphen{} The maximum number of successful steps in a given run where the new (smaller) objective value is larger than the best value found in a previous run. Default is \sphinxcode{\sphinxupquote{maxfun}}, the input to \sphinxcode{\sphinxupquote{dfols.solve()}}.

\item {} 
\sphinxAtStartPar
\sphinxcode{\sphinxupquote{restarts.auto\_detect}} \sphinxhyphen{} Whether or not to automatically determine when to restart. This is an extra condition, and restarts can still be triggered by small trust region radius, etc. Default is \sphinxcode{\sphinxupquote{True}}.

\item {} 
\sphinxAtStartPar
\sphinxcode{\sphinxupquote{restarts.auto\_detect.history}} \sphinxhyphen{} How many iterations of data on model changes and trust region radii to store. There are two criteria used: trust region radius decreases (no increases over the history, more decreases than no changes), and change in model Jacobian (consistently increasing trend as measured by slope and correlation coefficient of line of best fit). Default is 30.

\item {} 
\sphinxAtStartPar
\sphinxcode{\sphinxupquote{restarts.auto\_detect.min\_chgJ\_slope}} \sphinxhyphen{} Minimum rate of increase of \(\log(\|J_k-J_{k-1}\|_F)\) over the past iterations to cause a restart. Default is 0.015.

\item {} 
\sphinxAtStartPar
\sphinxcode{\sphinxupquote{restarts.auto\_detect.min\_correl}} \sphinxhyphen{} Minimum correlation of the data set \((k, \log(\|J_k-J_{k-1}\|_F))\) required to cause a restart. Default is 0.1.

\end{itemize}


\section{Dynamically Growing Initial Set}
\label{\detokenize{advanced:dynamically-growing-initial-set}}\begin{itemize}
\item {} 
\sphinxAtStartPar
\sphinxcode{\sphinxupquote{growing.ndirs\_initial}} \sphinxhyphen{} Number of initial points to add (excluding \(x_k\)). This should only be changed to a value less than \(n\), and only if the default setup cost of \(n+1\) evaluations of \sphinxcode{\sphinxupquote{objfun}} is impractical. If this is set to be less than the default, the input value \sphinxcode{\sphinxupquote{npt}} should be set to \(n\). If the default is used, all the below parameters have no effect on DFO\sphinxhyphen{}LS. Default is \sphinxcode{\sphinxupquote{npt\sphinxhyphen{}1}}.

\item {} 
\sphinxAtStartPar
\sphinxcode{\sphinxupquote{growing.full\_rank.use\_full\_rank\_interp}} \sphinxhyphen{} If \sphinxcode{\sphinxupquote{growing.ndirs\_initial}} is less than \sphinxcode{\sphinxupquote{npt}}, whether to perturb the interpolated \(J_k\) to make it full rank, allowing the trust region step to include components in the full search space. Default is \sphinxcode{\sphinxupquote{True}} if \(m\geq n\) and \sphinxcode{\sphinxupquote{False}} otherwise (opposite to \sphinxcode{\sphinxupquote{growing.perturb\_trust\_region\_step}}).

\item {} 
\sphinxAtStartPar
\sphinxcode{\sphinxupquote{growing.perturb\_trust\_region\_step}} \sphinxhyphen{} Whether to perturb the trust region step by an orthogonal direction not yet searched. This is an alternative to \sphinxcode{\sphinxupquote{growing.full\_rank.use\_full\_rank\_interp}}. Default is \sphinxcode{\sphinxupquote{False}} if \(m\geq n\) and \sphinxcode{\sphinxupquote{True}} otherwise (opposite to \sphinxcode{\sphinxupquote{growing.full\_rank.use\_full\_rank\_interp}}).

\item {} 
\sphinxAtStartPar
\sphinxcode{\sphinxupquote{growing.delta\_scale\_new\_dirns}} \sphinxhyphen{} When adding new search directions, the length of the step as a multiple of \(\Delta_k\). Default is 1, or 0.1 if \sphinxcode{\sphinxupquote{growing.perturb\_trust\_region\_step=True}}.

\item {} 
\sphinxAtStartPar
\sphinxcode{\sphinxupquote{growing.full\_rank.scale\_factor}} \sphinxhyphen{} Magnitude of extra components added to \(J_k\). Default is \(10^{-2}\).

\item {} 
\sphinxAtStartPar
\sphinxcode{\sphinxupquote{growing.full\_rank.svd\_scale\_factor}} \sphinxhyphen{} Floor singular values of \(J_k\) at this factor of the last nonzero value. Default is 1.

\item {} 
\sphinxAtStartPar
\sphinxcode{\sphinxupquote{growing.full\_rank.min\_sing\_val}} \sphinxhyphen{} Absolute floor on singular values of \(J_k\). Default is \(10^{-6}\).

\item {} 
\sphinxAtStartPar
\sphinxcode{\sphinxupquote{growing.full\_rank.svd\_max\_jac\_cond}} \sphinxhyphen{} Cap on condition number of \(J_k\) after applying floors to singular values (effectively another floor on the smallest singular value, since the largest singular value is fixed). Default is \(10^8\).

\item {} 
\sphinxAtStartPar
\sphinxcode{\sphinxupquote{growing.do\_geom\_steps}} \sphinxhyphen{} While still growing the initial set, whether to do geometry\sphinxhyphen{}improving steps in the trust region algorithm, as per the usual algorithm. Default is \sphinxcode{\sphinxupquote{False}}.

\item {} 
\sphinxAtStartPar
\sphinxcode{\sphinxupquote{growing.safety.do\_safety\_step}} \sphinxhyphen{} While still growing the initial set, whether to perform safety steps, or the regular trust region steps. Default is \sphinxcode{\sphinxupquote{True}}.

\item {} 
\sphinxAtStartPar
\sphinxcode{\sphinxupquote{growing.safety.reduce\_delta}} \sphinxhyphen{} While still growing the initial set, whether to reduce \(\Delta_k\) in safety steps. Default is \sphinxcode{\sphinxupquote{False}}.

\item {} 
\sphinxAtStartPar
\sphinxcode{\sphinxupquote{growing.safety.full\_geom\_step}} \sphinxhyphen{} While still growing the initial set, whether to do a full geometry\sphinxhyphen{}improving step within safety steps (the same as the post\sphinxhyphen{}growing phase of the algorithm). Since this involves reducing \(\Delta_k\), cannot be \sphinxcode{\sphinxupquote{True}} if \sphinxcode{\sphinxupquote{growing.safety.reduce\_delta}} is \sphinxcode{\sphinxupquote{True}}. Default is \sphinxcode{\sphinxupquote{False}}.

\item {} 
\sphinxAtStartPar
\sphinxcode{\sphinxupquote{growing.reset\_delta}} \sphinxhyphen{} Whether or not to reset trust region radius \(\Delta_k\) to its initial value at the end of the growing phase. Default is \sphinxcode{\sphinxupquote{False}}.

\item {} 
\sphinxAtStartPar
\sphinxcode{\sphinxupquote{growing.reset\_rho}} \sphinxhyphen{} Whether or not to reset trust region radius lower bound \(\rho_k\) to its initial value at the end of the growing phase. Default is \sphinxcode{\sphinxupquote{False}}.

\item {} 
\sphinxAtStartPar
\sphinxcode{\sphinxupquote{growing.gamma\_dec}} \sphinxhyphen{} Trust region decrease parameter during the growing phase. Default is \sphinxcode{\sphinxupquote{tr\_radius.gamma\_dec}}.

\item {} 
\sphinxAtStartPar
\sphinxcode{\sphinxupquote{growing.num\_new\_dirns\_each\_iter}} \sphinxhyphen{} Number of new search directions to add with each iteration where we do not have a full set of search directions. Default is 0, as this approach is not recommended.

\end{itemize}


\section{Dykstra’s Algorithm}
\label{\detokenize{advanced:dykstra-s-algorithm}}\begin{itemize}
\item {} 
\sphinxAtStartPar
\sphinxcode{\sphinxupquote{dykstra.d\_tol}} \sphinxhyphen{} Tolerance on the stopping conditions of Dykstra’s algorithm. Default is \(10^{-10}\).

\item {} 
\sphinxAtStartPar
\sphinxcode{\sphinxupquote{dykstra.max\_iters}} \sphinxhyphen{} The maximum number of iterations Dykstra’s algorithm is allowed to take before stopping. Default is \(100\).

\end{itemize}


\section{Checking Matrix Rank}
\label{\detokenize{advanced:checking-matrix-rank}}\begin{itemize}
\item {} 
\sphinxAtStartPar
\sphinxcode{\sphinxupquote{matrix\_rank.r\_tol}} \sphinxhyphen{} Tolerance on what is the smallest posisble diagonal entry value in the QR factorization before being considered zero. Default is \(10^{-18}\).

\end{itemize}


\section{Handling regularizer}
\label{\detokenize{advanced:handling-regularizer}}\begin{itemize}
\item {} 
\sphinxAtStartPar
\sphinxcode{\sphinxupquote{func\_tol.criticality\_measure}} \sphinxhyphen{} scale factor (of the current trust\sphinxhyphen{}region radius) to determine the accuracy of the calculated  criticality/stationarity measure (smaller means more accurate). Default is \(10^{-3}\).

\item {} 
\sphinxAtStartPar
\sphinxcode{\sphinxupquote{func\_tol.tr\_step}} \sphinxhyphen{} scale factor to determine the accuracy of the trust\sphinxhyphen{}region step (smaller is less accurate). Default is \(0.9\).

\item {} 
\sphinxAtStartPar
\sphinxcode{\sphinxupquote{func\_tol.max\_iters}} \sphinxhyphen{} maximum number of subproblem (S\sphinxhyphen{}FISTA) iterations. Default is 500.

\item {} 
\sphinxAtStartPar
\sphinxcode{\sphinxupquote{sfista.max\_iters\_scaling}} \sphinxhyphen{} by what factor to increase the minimum number of subproblem (S\sphinxhyphen{}FISTA) iterations. Must be at least 1. Default is 2.

\end{itemize}


\section{References}
\label{\detokenize{advanced:references}}
\sphinxstepscope


\chapter{Diagnostic Information}
\label{\detokenize{diagnostic:diagnostic-information}}\label{\detokenize{diagnostic::doc}}
\sphinxAtStartPar
In {\hyperref[\detokenize{userguide::doc}]{\sphinxcrossref{\DUrole{doc}{Using DFO\sphinxhyphen{}LS}}}}, we saw that the output of DFO\sphinxhyphen{}LS returns a container which includes diagnostic information about the progress of the algorithm (\sphinxcode{\sphinxupquote{soln.diagnostic\_info}}). This object is a \sphinxhref{http://pandas.pydata.org/}{Pandas} DataFrame, with one row per iteration of the algorithm. In this section, we explain the meaning of each type of output (the columns of the DataFrame).

\sphinxAtStartPar
To save this information to a CSV file, use:
\begin{quote}

\begin{sphinxVerbatim}[commandchars=\\\{\}]
\PYG{c+c1}{\PYGZsh{} Previously: define objfun and x0}

\PYG{c+c1}{\PYGZsh{} Turn on diagnostic information}
\PYG{n}{user\PYGZus{}params} \PYG{o}{=} \PYG{p}{\PYGZob{}}\PYG{l+s+s1}{\PYGZsq{}}\PYG{l+s+s1}{logging.save\PYGZus{}diagnostic\PYGZus{}info}\PYG{l+s+s1}{\PYGZsq{}}\PYG{p}{:} \PYG{k+kc}{True}\PYG{p}{\PYGZcb{}}

\PYG{c+c1}{\PYGZsh{} Call DFO\PYGZhy{}LS}
\PYG{n}{soln} \PYG{o}{=} \PYG{n}{dfols}\PYG{o}{.}\PYG{n}{solve}\PYG{p}{(}\PYG{n}{objfun}\PYG{p}{,} \PYG{n}{x0}\PYG{p}{,} \PYG{n}{user\PYGZus{}params}\PYG{o}{=}\PYG{n}{user\PYGZus{}params}\PYG{p}{)}

\PYG{c+c1}{\PYGZsh{} Save diagnostic info to CSV}
\PYG{n}{soln}\PYG{o}{.}\PYG{n}{diagnostic\PYGZus{}info}\PYG{o}{.}\PYG{n}{to\PYGZus{}csv}\PYG{p}{(}\PYG{l+s+s1}{\PYGZsq{}}\PYG{l+s+s1}{myfile.csv}\PYG{l+s+s1}{\PYGZsq{}}\PYG{p}{)}
\end{sphinxVerbatim}
\end{quote}

\sphinxAtStartPar
Depending on exactly how DFO\sphinxhyphen{}LS terminates, the last row of results may not be fully populated.


\section{Current Iterate}
\label{\detokenize{diagnostic:current-iterate}}\begin{itemize}
\item {} 
\sphinxAtStartPar
\sphinxcode{\sphinxupquote{xk}} \sphinxhyphen{} Best point found so far (current iterate). This is only saved if \sphinxcode{\sphinxupquote{user\_params{[}\textquotesingle{}logging.save\_xk\textquotesingle{}{]} = True}}.

\item {} 
\sphinxAtStartPar
\sphinxcode{\sphinxupquote{rk}} \sphinxhyphen{} The vector of residuals at the current iterate. This is only saved if \sphinxcode{\sphinxupquote{user\_params{[}\textquotesingle{}logging.save\_rk\textquotesingle{}{]} = True}}.

\item {} 
\sphinxAtStartPar
\sphinxcode{\sphinxupquote{fk}} \sphinxhyphen{} The value of \(f\) at the current iterate.

\end{itemize}


\section{Trust Region}
\label{\detokenize{diagnostic:trust-region}}\begin{itemize}
\item {} 
\sphinxAtStartPar
\sphinxcode{\sphinxupquote{rho}} \sphinxhyphen{} The lower bound on the trust region radius \(\rho_k\).

\item {} 
\sphinxAtStartPar
\sphinxcode{\sphinxupquote{delta}} \sphinxhyphen{} The trust region radius \(\Delta_k\).

\item {} 
\sphinxAtStartPar
\sphinxcode{\sphinxupquote{norm\_sk}} \sphinxhyphen{} The norm of the trust region step \(\|s_k\|\).

\end{itemize}


\section{Model Interpolation}
\label{\detokenize{diagnostic:model-interpolation}}\begin{itemize}
\item {} 
\sphinxAtStartPar
\sphinxcode{\sphinxupquote{npt}} \sphinxhyphen{} The number of interpolation points.

\item {} 
\sphinxAtStartPar
\sphinxcode{\sphinxupquote{interpolation\_error}} \sphinxhyphen{} The sum of squares of the interpolation errors from the interpolated model.

\item {} 
\sphinxAtStartPar
\sphinxcode{\sphinxupquote{interpolation\_condition\_number}} \sphinxhyphen{} The condition number of the matrix in the interpolation linear system.

\item {} 
\sphinxAtStartPar
\sphinxcode{\sphinxupquote{interpolation\_change\_J\_norm}} \sphinxhyphen{} The Frobenius norm of the change in Jacobian at this iteration, \(\|J_k-J_{k-1}\|_F\).

\item {} 
\sphinxAtStartPar
\sphinxcode{\sphinxupquote{interpolation\_total\_residual}} \sphinxhyphen{} The total residual from the interpolation optimization problem.

\item {} 
\sphinxAtStartPar
\sphinxcode{\sphinxupquote{poisedness}} \sphinxhyphen{} The smallest value of \(\Lambda\) for which the current interpolation set \(Y_k\) is \(\Lambda\)\sphinxhyphen{}poised in the current trust region. This is the most expensive piece of information to compute, and is only computed if \sphinxcode{\sphinxupquote{user\_params{[}\textquotesingle{}logging.save\_poisedness\textquotesingle{} = True}}.

\item {} 
\sphinxAtStartPar
\sphinxcode{\sphinxupquote{max\_distance\_xk}} \sphinxhyphen{} The maximum distance from any interpolation point to the current iterate.

\item {} 
\sphinxAtStartPar
\sphinxcode{\sphinxupquote{norm\_gk}} \sphinxhyphen{} The norm of the model gradient \(\|g_k\|\).

\end{itemize}


\section{Iteration Count}
\label{\detokenize{diagnostic:iteration-count}}\begin{itemize}
\item {} 
\sphinxAtStartPar
\sphinxcode{\sphinxupquote{nruns}} \sphinxhyphen{} The number of times the algorithm has been restarted.

\item {} 
\sphinxAtStartPar
\sphinxcode{\sphinxupquote{nf}} \sphinxhyphen{} The number of objective evaluations so far (see \sphinxcode{\sphinxupquote{soln.nf}})

\item {} 
\sphinxAtStartPar
\sphinxcode{\sphinxupquote{nx}} \sphinxhyphen{} The number of points at which the objective has been evaluated so far (see \sphinxcode{\sphinxupquote{soln.nx}})

\item {} 
\sphinxAtStartPar
\sphinxcode{\sphinxupquote{nsamples}} \sphinxhyphen{} The total number of objective evaluations used for all current interpolation points.

\item {} 
\sphinxAtStartPar
\sphinxcode{\sphinxupquote{iter\_this\_run}} \sphinxhyphen{} The number of iterations since the last restart.

\item {} 
\sphinxAtStartPar
\sphinxcode{\sphinxupquote{iters\_total}} \sphinxhyphen{} The total number of iterations so far.

\end{itemize}


\section{Algorithm Progress}
\label{\detokenize{diagnostic:algorithm-progress}}\begin{itemize}
\item {} 
\sphinxAtStartPar
\sphinxcode{\sphinxupquote{iter\_type}} \sphinxhyphen{} A text description of what type of iteration we had (e.g. Successful, Safety, etc.)

\item {} 
\sphinxAtStartPar
\sphinxcode{\sphinxupquote{ratio}} \sphinxhyphen{} The ratio of actual to predicted objective reduction in the trust region step.

\item {} 
\sphinxAtStartPar
\sphinxcode{\sphinxupquote{slow\_iter}} \sphinxhyphen{} Equal to 1 if the current iteration is successful but slow, 0 if is successful but not slow, and \sphinxhyphen{}1 if was not successful.

\end{itemize}

\sphinxstepscope


\chapter{Version History}
\label{\detokenize{history:version-history}}\label{\detokenize{history::doc}}
\sphinxAtStartPar
This section lists the different versions of DFO\sphinxhyphen{}LS and the updates between them.


\section{Version 1.0 (6 Feb 2018)}
\label{\detokenize{history:version-1-0-6-feb-2018}}\begin{itemize}
\item {} 
\sphinxAtStartPar
Initial release of DFO\sphinxhyphen{}LS

\end{itemize}


\section{Version 1.0.1 (20 Feb 2018)}
\label{\detokenize{history:version-1-0-1-20-feb-2018}}\begin{itemize}
\item {} 
\sphinxAtStartPar
Minor bug fix to trust region subproblem solver (the output \sphinxcode{\sphinxupquote{crvmin}} is calculated correctly) \sphinxhyphen{} this has minimal impact on the performance of DFO\sphinxhyphen{}LS.

\end{itemize}


\section{Version 1.0.2 (20 Jun 2018)}
\label{\detokenize{history:version-1-0-2-20-jun-2018}}\begin{itemize}
\item {} 
\sphinxAtStartPar
Extra optional input \sphinxcode{\sphinxupquote{args}} which passes through arguments for \sphinxcode{\sphinxupquote{objfun}}.

\item {} 
\sphinxAtStartPar
Bug fixes: default parameters for reduced initialization cost regime, returning correct value if exiting from within a safety step, retrieving dependencies during installation.

\end{itemize}


\section{Version 1.1 (16 Jan 2019)}
\label{\detokenize{history:version-1-1-16-jan-2019}}\begin{itemize}
\item {} 
\sphinxAtStartPar
Use different default reduced initialization cost method for inverse problems to ensure whole space is searched correctly.

\item {} 
\sphinxAtStartPar
Bug fixes: default trust region radius when scaling feasible region, exit correctly when no Jacobian returned, handling overflow at initial value

\end{itemize}


\section{Version 1.1.1 (5 Apr 2019)}
\label{\detokenize{history:version-1-1-1-5-apr-2019}}\begin{itemize}
\item {} 
\sphinxAtStartPar
Link code to Zenodo, to create DOI \sphinxhyphen{} no changes to the DFO\sphinxhyphen{}LS algorithm.

\end{itemize}


\section{Version 1.2 (12 Feb 2020)}
\label{\detokenize{history:version-1-2-12-feb-2020}}\begin{itemize}
\item {} 
\sphinxAtStartPar
Use deterministic initialisation by default (so it is no longer necessary to set a random seed for reproducibility of DFO\sphinxhyphen{}LS results).

\item {} 
\sphinxAtStartPar
Full model Hessian stored rather than just upper triangular part \sphinxhyphen{} this improves the runtime of Hessian\sphinxhyphen{}based operations.

\item {} 
\sphinxAtStartPar
Faster trust\sphinxhyphen{}region and geometry subproblem solutions in Fortran using the \sphinxhref{https://github.com/lindonroberts/trust-region}{trustregion} package.

\item {} 
\sphinxAtStartPar
Faster interpolation solution for multiple right\sphinxhyphen{}hand sides.

\item {} 
\sphinxAtStartPar
Don’t adjust starting point if it is close to the bounds (as long as it is feasible).

\item {} 
\sphinxAtStartPar
Option to stop default logging behavior and/or enable per\sphinxhyphen{}iteration printing.

\item {} 
\sphinxAtStartPar
Bugfix: correctly handle 1\sphinxhyphen{}sided bounds as inputs, avoid divide\sphinxhyphen{}by\sphinxhyphen{}zero warnings when auto\sphinxhyphen{}detecting restarts.

\end{itemize}


\section{Version 1.2.1 (13 Feb 2020)}
\label{\detokenize{history:version-1-2-1-13-feb-2020}}\begin{itemize}
\item {} 
\sphinxAtStartPar
Make the use of the \sphinxhref{https://github.com/lindonroberts/trust-region}{trustregion} package optional, not installed by default.

\end{itemize}


\section{Version 1.2.2 (26 Feb 2021)}
\label{\detokenize{history:version-1-2-2-26-feb-2021}}\begin{itemize}
\item {} 
\sphinxAtStartPar
Minor update to remove NumPy deprecation warnings \sphinxhyphen{} no changes to the DFO\sphinxhyphen{}LS algorithm.

\end{itemize}


\section{Version 1.2.3 (1 Jun 2021)}
\label{\detokenize{history:version-1-2-3-1-jun-2021}}\begin{itemize}
\item {} 
\sphinxAtStartPar
Minor update to customise handling of NaNs in objective evaluations \sphinxhyphen{} no changes to the DFO\sphinxhyphen{}LS algorithm.

\end{itemize}


\section{Version 1.3.0 (8 Nov 2021)}
\label{\detokenize{history:version-1-3-0-8-nov-2021}}\begin{itemize}
\item {} 
\sphinxAtStartPar
Handle finitely many arbitrary convex constraints in addition to simple bound constraints.

\item {} 
\sphinxAtStartPar
Add module\sphinxhyphen{}level logging for more informative log outputs.

\item {} 
\sphinxAtStartPar
Only new functionality is added, so there is no change to the solver for unconstrained/bound\sphinxhyphen{}constrained problems.

\end{itemize}


\section{Version 1.4.0 (29 Jan 2024)}
\label{\detokenize{history:version-1-4-0-29-jan-2024}}\begin{itemize}
\item {} 
\sphinxAtStartPar
Require newer SciPy version (at least 1.11) as a dependency \sphinxhyphen{} avoids occasional undetermined behavior but no changes to the DFO\sphinxhyphen{}LS algorithm.

\item {} 
\sphinxAtStartPar
Gracefully handle NaN objective value from evaluation at a new trial point (trust\sphinxhyphen{}region step).

\end{itemize}


\section{Version 1.4.1 (11 Apr 2024)}
\label{\detokenize{history:version-1-4-1-11-apr-2024}}\begin{itemize}
\item {} 
\sphinxAtStartPar
Migrate package setup to pyproject.toml (required for Python version 3.12)

\item {} 
\sphinxAtStartPar
Drop support for Python 2.7 and \textless{}=3.8 in line with SciPy \textgreater{}=1.11 dependency (introduced v1.4.0)

\end{itemize}


\section{Version 1.5.0 (11 Sep 2024)}
\label{\detokenize{history:version-1-5-0-11-sep-2024}}\begin{itemize}
\item {} 
\sphinxAtStartPar
Add support for (possibly nonsmooth) regularizer term.

\item {} 
\sphinxAtStartPar
Drop warning about infeasible initial point if the point is on an upper/lower bound.

\end{itemize}


\section{Version 1.5.1 (10 Oct 2024)}
\label{\detokenize{history:version-1-5-1-10-oct-2024}}\begin{itemize}
\item {} 
\sphinxAtStartPar
Add return values \sphinxcode{\sphinxupquote{soln.xmin\_eval\_num}} and \sphinxcode{\sphinxupquote{soln.jacmin\_eval\_nums}}

\item {} 
\sphinxAtStartPar
Allow option for parallel initial evaluations for non\sphinxhyphen{}random directions if \sphinxcode{\sphinxupquote{npt}} not too large

\end{itemize}


\section{Version 1.5.2 (28 Oct 2024)}
\label{\detokenize{history:version-1-5-2-28-oct-2024}}\begin{itemize}
\item {} 
\sphinxAtStartPar
Bugfix for saving diagnostic info (bug introduced in v1.5.1)

\end{itemize}


\section{Version 1.5.3 (30 Oct 2024)}
\label{\detokenize{history:version-1-5-3-30-oct-2024}}\begin{itemize}
\item {} 
\sphinxAtStartPar
Bugfix when starting solver at problem minimizer (bug introduced in v1.5.1)

\end{itemize}

\sphinxstepscope


\chapter{Contributors}
\label{\detokenize{contributors:contributors}}\label{\detokenize{contributors::doc}}

\section{Main author}
\label{\detokenize{contributors:main-author}}\begin{itemize}
\item {} 
\sphinxAtStartPar
\sphinxhref{https://lindonroberts.github.io/}{Lindon Roberts} (University of Sydney)

\end{itemize}


\section{Contributors}
\label{\detokenize{contributors:id1}}\begin{itemize}
\item {} 
\sphinxAtStartPar
Matthew Hough (University of Waterloo): handle general convex constraints {[}version 1.3{]}

\item {} 
\sphinxAtStartPar
Yanjun Liu (Princeton University): nonsmooth regularizer {[}version 1.5{]}

\item {} 
\sphinxAtStartPar
Kevin Lam (Australian National University): nonsmooth regularizer {[}version 1.5{]}

\end{itemize}


\chapter{Acknowledgements}
\label{\detokenize{index:acknowledgements}}
\sphinxAtStartPar
This software was initially developed under the supervision of \sphinxhref{https://www.maths.ox.ac.uk/people/coralia.cartis}{Coralia Cartis}, and was supported by the EPSRC Centre For Doctoral Training in \sphinxhref{https://www.maths.ox.ac.uk/study-here/postgraduate-study/industrially-focused-mathematical-modelling-epsrc-cdt}{Industrially Focused Mathematical Modelling} (EP/L015803/1) in collaboration with the \sphinxhref{http://www.nag.com/}{Numerical Algorithms Group}. Development of DFO\sphinxhyphen{}LS has also been supported by the Australian Research Council (DE240100006).

\begin{sphinxthebibliography}{CFMR2018}
\bibitem[CFMR2018]{info:cfmr2018}
\sphinxAtStartPar
Coralia Cartis, Jan Fiala, Benjamin Marteau and Lindon Roberts, \sphinxhref{https://doi.org/10.1145/3338517}{Improving the Flexibility and Robustness of Model\sphinxhyphen{}Based Derivative\sphinxhyphen{}Free Optimization Solvers}, \sphinxstyleemphasis{ACM Transactions on Mathematical Software}, 45:3 (2019), pp. 32:1\sphinxhyphen{}32:41 {[}\sphinxhref{https://arxiv.org/abs/1804.00154}{preprint}{]}
\bibitem[HR2022]{info:hr2022}
\sphinxAtStartPar
Matthew Hough and Lindon Roberts, \sphinxhref{https://doi.org/10.1137/21M1460971}{Model\sphinxhyphen{}Based Derivative\sphinxhyphen{}Free Methods for Convex\sphinxhyphen{}Constrained Optimization}, \sphinxstyleemphasis{SIAM Journal on Optimization}, 21:4 (2022), pp. 2552\sphinxhyphen{}2579 {[}\sphinxhref{https://arxiv.org/abs/2111.05443}{preprint}{]}.
\bibitem[LLR2024]{info:llr2024}
\sphinxAtStartPar
Yanjun Liu, Kevin H. Lam and Lindon Roberts, \sphinxhref{http://arxiv.org/abs/2407.14915}{Black\sphinxhyphen{}box Optimization Algorithms for Regularized Least\sphinxhyphen{}squares Problems}, \sphinxstyleemphasis{arXiv preprint arXiv:2407.14915} (2024).
\bibitem[CFMR2018]{userguide:cfmr2018}
\sphinxAtStartPar
Coralia Cartis, Jan Fiala, Benjamin Marteau and Lindon Roberts, \sphinxhref{https://doi.org/10.1145/3338517}{Improving the Flexibility and Robustness of Model\sphinxhyphen{}Based Derivative\sphinxhyphen{}Free Optimization Solvers}, \sphinxstyleemphasis{ACM Transactions on Mathematical Software}, 45:3 (2019), pp. 32:1\sphinxhyphen{}32:41 {[}\sphinxhref{https://arxiv.org/abs/1804.00154}{preprint}{]}
\bibitem[HR2022]{userguide:hr2022}
\sphinxAtStartPar
Matthew Hough and Lindon Roberts, \sphinxhref{https://doi.org/10.1137/21M1460971}{Model\sphinxhyphen{}Based Derivative\sphinxhyphen{}Free Methods for Convex\sphinxhyphen{}Constrained Optimization}, \sphinxstyleemphasis{SIAM Journal on Optimization}, 21:4 (2022), pp. 2552\sphinxhyphen{}2579 {[}\sphinxhref{https://arxiv.org/abs/2111.05443}{preprint}{]}.
\bibitem[LLR2024]{userguide:llr2024}
\sphinxAtStartPar
Yanjun Liu, Kevin H. Lam and Lindon Roberts, \sphinxhref{http://arxiv.org/abs/2407.14915}{Black\sphinxhyphen{}box Optimization Algorithms for Regularized Least\sphinxhyphen{}squares Problems}, \sphinxstyleemphasis{arXiv preprint arXiv:2407.14915} (2024).
\bibitem[B2017]{userguide:b2017}
\sphinxAtStartPar
Amir Beck, \sphinxhref{https://doi.org/10.1137/1.9781611974997}{First\sphinxhyphen{}Order Methods in Optimization}, SIAM (2017).
\bibitem[CFMR2018]{advanced:cfmr2018}
\sphinxAtStartPar
Coralia Cartis, Jan Fiala, Benjamin Marteau and Lindon Roberts, \sphinxhref{https://doi.org/10.1145/3338517}{Improving the Flexibility and Robustness of Model\sphinxhyphen{}Based Derivative\sphinxhyphen{}Free Optimization Solvers}, \sphinxstyleemphasis{ACM Transactions on Mathematical Software}, 45:3 (2019), pp. 32:1\sphinxhyphen{}32:41 {[}\sphinxhref{https://arxiv.org/abs/1804.00154}{preprint}{]}
\bibitem[HR2022]{advanced:hr2022}
\sphinxAtStartPar
Matthew Hough and Lindon Roberts, \sphinxhref{https://doi.org/10.1137/21M1460971}{Model\sphinxhyphen{}Based Derivative\sphinxhyphen{}Free Methods for Convex\sphinxhyphen{}Constrained Optimization}, \sphinxstyleemphasis{SIAM Journal on Optimization}, 21:4 (2022), pp. 2552\sphinxhyphen{}2579 {[}\sphinxhref{https://arxiv.org/abs/2111.05443}{preprint}{]}.
\bibitem[LLR2024]{advanced:llr2024}
\sphinxAtStartPar
Yanjun Liu, Kevin H. Lam and Lindon Roberts, \sphinxhref{http://arxiv.org/abs/2407.14915}{Black\sphinxhyphen{}box Optimization Algorithms for Regularized Least\sphinxhyphen{}squares Problems}, \sphinxstyleemphasis{arXiv preprint arXiv:2407.14915} (2024).
\end{sphinxthebibliography}



\renewcommand{\indexname}{Index}
\printindex
\end{document}